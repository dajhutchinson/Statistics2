\documentclass[11pt,a4paper]{article}

\usepackage[margin=1in, paperwidth=8.3in, paperheight=11.7in]{geometry}
\usepackage{amsfonts}
\usepackage{amsmath}
\usepackage{amssymb}
\usepackage{dsfont}
\usepackage{enumerate}
\usepackage{enumitem}
\usepackage{fancyhdr}
\usepackage{graphicx}
\usepackage{tikz}
\usepackage{changepage} 

\begin{document}

\pagestyle{fancy}
\setlength\parindent{0pt}
\allowdisplaybreaks

\renewcommand{\headrulewidth}{0pt}

% Cover page title
\title{Statistics 2 - Problem Sheet 3}
\author{Dom Hutchinson}
\maketitle

% Header
\fancyhead[L]{Dom Hutchinson}
\fancyhead[C]{Statistics 2 - Problem Sheet 3}
\fancyhead[R]{\today}

% Counters
\newcounter{qpart}[section]

% commands
\newcommand{\dotprod}[0]{\boldsymbol{\cdot}}
\newcommand{\cosech}[0]{\mathrm{cosech}\ }
\newcommand{\cosec}[0]{\mathrm{cosec}\ }
\newcommand{\sech}[0]{\mathrm{sech}\ }
\newcommand{\prob}[0]{\mathbb{P}}
\newcommand{\nats}[0]{\mathbb{N}}
\newcommand{\cov}[0]{\mathrm{cov}}
\newcommand{\var}[0]{\mathrm{var}}
\newcommand{\expect}[0]{\mathbb{E}}
\newcommand{\reals}[0]{\mathbb{R}}
\newcommand{\integers}[0]{\mathbb{Z}}
\newcommand{\indicator}[0]{\mathds{1}}
\newcommand{\nb}[0]{\textit{N.B.} }
\newcommand{\ie}[0]{\textit{i.e.} }
\newcommand{\eg}[0]{\textit{e.g.} }
\newcommand{\iid}[0]{\overset{\text{iid}}{\sim} }

\newcommand{\qpart}[0]{\stepcounter{qpart} \textbf{Question \arabic{section}.\arabic{qpart}\\}}
\newcommand{\qpartnb}[0]{\stepcounter{qpart} \textbf{Question \arabic{section}.\arabic{qpart}} - }
\newcommand{\ans}[0]{ \textbf{Answer \arabic{section}\\}}
\newcommand{\apart}[0]{ \textbf{Answer \arabic{section}.\arabic{qpart}\\}}
\newcommand{\apartnb}[0]{\stepcounter{qpart} \textbf{Answer \arabic{section}.\arabic{qpart}} - }
\newcommand{\question}[0]{\stepcounter{section}\section*{Question - \arabic{section}.}}

\question
This question is about maximum likelihood estimates of transformations of the parameter.\\

\qpart
Suppose $\textbf{X}\iid\text{Exponential}(\lambda)$ where $\lambda\in\Theta=\reals^{>0}$.\\
Consider the transformed parameter $\tau=\frac{1}{1+\lambda}$.\\
Write the likelihood function in terms of $\tau$, then maximise it to find $\hat{\tau}_{ml}$.\\
Confirm that $\hat{\tau}_{mle}=\frac{1}{1+\hat{\lambda}_{ml}}$.\\

\apart
We have $f_{X_i}(x_i;\lambda)=\lambda e^{-\lambda x_i}\ \forall\ i\in[1,n]$ since $\lambda>0$.
\[\begin{array}{rrcl}
&\tau&=&\frac{1}{1+\lambda}\\
\implies&1+\lambda&=&\frac{1}{\tau}\\
\implies&\lambda&=&\frac{1}{\tau}-1=\frac{1-\tau}{\tau}\\
&L(\lambda;\textbf{x})&=&\prod\limits_{i=1}^nf_{X_i}(x_i;\lambda)\\
&&=&\prod\limits_{i=1}^n\lambda e^{-\lambda x_i}\\
&&=&\lambda^ne^{-\lambda\sum_{i=1}^nx_i}\\
\implies&L(\tau;\textbf{x})&=&\left(\frac{1-\tau}{\tau}\right)^ne^{\frac{\tau-1}{\tau}\sum_{i=1}^nx_i}\\
\implies&\ell(\tau;\textbf{x})&=&n\ln(1-\tau)-n\ln\tau+\frac{\tau-1}{\tau}\sum\limits_{i=1}^nx_i\\
\implies&\frac{\partial}{\partial\tau}\ell(\tau;\textbf{x})&=&-\frac{n}{1-\tau}-\frac{n}{\tau}+\frac{\tau-(\tau-1)}{\tau^2}\sum\limits_{i=1}^nx_i\\
&&=&-\frac{n}{1-\tau}-\frac{n}{\tau}+\frac{1}{\tau^2}\sum\limits_{i=1}^nx_i\\
\text{Setting}&0&=&\frac{\partial}{\partial\tau}\ell(\hat{\tau};\textbf{x})\\
\implies&0&=&-\frac{n}{1-\hat{\tau}}-\frac{n}{\hat{\tau}}+\frac{1}{\hat{\tau}^2}\sum\limits_{i=1}^nx_i\\
\implies&0&=&-n\hat{\tau}^2-n\hat{\tau}(1-\hat{\tau})+(1-\hat{\tau})\sum\limits_{i=1}^nx_i\\
&&=&-n\hat{\tau}^2-n\hat{\tau}+n\hat{\tau}^2+\sum\limits_{i=1}^nx_i-\hat{\tau}\sum\limits_{i=1}^nx_i\\
\implies&\sum\limits_{i=1}^nx_i&=&\hat{\tau}\left(n+\sum\limits_{i=1}^nx_i\right)\\
\implies&\hat{\tau}&=&\dfrac{\sum_{i=1}^nx_i}{n+\sum_{i=1}^nx_i}\\
&&=&\dfrac{1}{1+\frac{n}{\sum_{i=1}^nx_i}}\\
\text{We have}&L(\lambda;\textbf{x})&=&\lambda^ne^{-\lambda\sum_{i=1}^nx_i}\\
\implies&\ell(\lambda;\textbf{x})&=&n\ln\lambda-\lambda\sum\limits_{i=1}^nx_i\\
\implies&\frac{\partial}{\partial\lambda}\ell(\lambda;\textbf{x})&=&\frac{n}{\lambda}-\sum\limits_{i=1}^nx_i\\
\text{Setting}&0&=&\frac{\partial}{\partial\hat{\lambda}}\ell(\lambda;\textbf{x})\\
&&=&\dfrac{n}{\hat{\lambda}}-\sum\limits_{i=1}^nx_i\\
\implies&\hat{\lambda}&=&\dfrac{n}{\sum_{i=1}^nx_i}\\
\implies&\hat{\tau}&=&\dfrac{1}{1+\hat{\lambda}}
\end{array}\]

\qpart
If $X\sim\text{Poisson}(\lambda)$ where $\lambda\in\Theta=\reals^{>0}$ then $\prob(X=0;\lambda)=e^{-\lambda}$.\\
Explain why the maximum likelihood estimate of this probability is $e^{-\hat{\lambda}_{ml}}$.\\

\apart
%TODO fluff this up
Define $\tau(\lambda)=e^{-\lambda}$. Then $\hat{\tau}(\lambda)=e^{-\hat{\lambda}}$.

\question
Let $X_1,X_2,\dots\iid\text{Uniform}[0,1]$ each with probability density function $f(x)=\mathds{1}\{0\leq x\leq1\}$.\\

\qpartnb Calculate the mean and variance of $\ln(X_1)$.\\

\apart
%TODO expand on how I have done integration by parts
\[\begin{array}{rcl}
\expect(\ln(X_1))&=&{\displaystyle\int f(x)\ln(x)dx}\\
&=&{\displaystyle\int\mathds{1}\{0\leq x\leq1\}ln(x)dx}\\
&=&{\displaystyle\int_0^1\ln(x)dx}\\
&=&[x\ln(x)-x]_0^1\\
&=&(1\ln(1)-1)-(0\ln(0)-0)\\
&=&(-1)-(0)\\
&=&-1\\
\var(\ln(X_1))&=&\expect(\ln(X_1)^2)-\expect(\ln(X_1))^2\\
&=&{\displaystyle\int f(x)\ln(x)^2dx-1}\\
&=&{\displaystyle\int \mathds{1}\{0\leq x\leq1\}\ln(x)^2dx-1}\\
&=&{\displaystyle\int_0^1\ln(x)^2dx-1}\\
&=&{[x\ln(x)^2]_0^1-\displaystyle\int_0^12\ln(x)dx-1}\\
&=&[x\ln(x)^2-2(x\ln(x)-x)]_0^1-1\\
&=&(\ln(1)^2-2\ln(1)+2)-0-1\\
&=&2-1\\
&=&1
\end{array}\]


\qpartnb By taking logs, find a random variable $X$ such that as $n\to\infty$
$$(X_1,\dots,X_n)^{\frac{1}{\sqrt{n}}}e^{\sqrt{n}}\to_\mathcal{D}X$$

\apart
%TODO

\end{document}