\documentclass[11pt,a4paper]{article}

\usepackage[margin=1in, paperwidth=8.3in, paperheight=11.7in]{geometry}
\usepackage{amsfonts}
\usepackage{amsmath}
\usepackage{amssymb}
\usepackage{dsfont}
\usepackage{enumerate}
\usepackage{enumitem}
\usepackage{fancyhdr}
\usepackage{graphicx}
\usepackage{tikz}
\usepackage{changepage} 

\begin{document}

\pagestyle{fancy}
\setlength\parindent{0pt}
\allowdisplaybreaks

\renewcommand{\headrulewidth}{0pt}

% Cover page title
\title{Statistics 2 - Problem Sheet 4}
\author{Dom Hutchinson}
\maketitle

% Header
\fancyhead[L]{Dom Hutchinson}
\fancyhead[C]{Statistics 2 - Problem Sheet 4}
\fancyhead[R]{\today}

% Counters
\newcounter{qpart}[section]

% commands
\newcommand{\dotprod}[0]{\boldsymbol{\cdot}}
\newcommand{\cosech}[0]{\mathrm{cosech}\ }
\newcommand{\cosec}[0]{\mathrm{cosec}\ }
\newcommand{\sech}[0]{\mathrm{sech}\ }
\newcommand{\prob}[0]{\mathbb{P}}
\newcommand{\nats}[0]{\mathbb{N}}
\newcommand{\cov}[0]{\mathrm{Cov}}
\newcommand{\var}[0]{\mathrm{Var}}
\newcommand{\expect}[0]{\mathbb{E}}
\newcommand{\reals}[0]{\mathbb{R}}
\newcommand{\integers}[0]{\mathbb{Z}}
\newcommand{\indicator}[0]{\mathds{1}}
\newcommand{\nb}[0]{\textit{N.B.} }
\newcommand{\ie}[0]{\textit{i.e.} }
\newcommand{\eg}[0]{\textit{e.g.} }
\newcommand{\iid}[0]{\overset{\text{iid}}{\sim} }

\newcommand{\qpart}[0]{\stepcounter{qpart} \textbf{Question \arabic{section}.\arabic{qpart}\\}}
\newcommand{\qpartt}[0]{\stepcounter{qpart} \textbf{Question \arabic{section}.\arabic{qpart}} - }
\newcommand{\ans}[0]{ \textbf{Answer \arabic{section}\\}}
\newcommand{\apart}[0]{ \textbf{Answer \arabic{section}.\arabic{qpart}\\}}
\newcommand{\apartt}[0]{\textbf{Answer \arabic{section}.\arabic{qpart}} - }
\newcommand{\question}[0]{\stepcounter{section}\section*{Question - \arabic{section}.}}

\question
For the following statistical models derive the score function, $\ell'(\theta;x)$, and the Fisher Information, $I(\theta)$. Assume each model to be regular.\\

\qpartt
$X\sim\text{Geometric}(p)$\\

\apart
Let $X\sim\text{Geometric}(p)$ with $p$ unknown. Then
\[\begin{array}{rrcl}
\text{We have}&L(p;x)&\propto&f(x;p)\\
&&=&(1-p)^{x-1}p\\
\implies&\ell(p;x)&=&\ln f(x;p)\\
&&=&(x-1)\ln(1-p)+\ln p+c\\
\text{By definition}&\ell'(p;x)&=&\frac{d}{dp}\ell(p;x)\\
&&=&-\frac{x-1}{1-p}+\frac{1}{p}\\
&&=&\frac{1-x}{1-p}+\frac{1}{p}\\
\text{By definition}&I(p)&=&\expect(\ell'(p;X)^2;p)\\
&&=&\var(\ell'(p;X);p)\text{ by regularity conditions}\\
&&=&\var\left(\frac{1-X}{1-p}+\frac{1}{p};p\right)\\
&&=&\var\left(\frac{1-X}{1-p};p\right)\\
&&=&\frac{1}{(1-p)^2}\var\left(1-X;p\right)\\
&&=&\frac{1}{(1-p)^2}\var\left(X;p\right)\\
&&=&\frac{1}{(1-p)^2}\times\frac{1-p}{p^2}\\
&&=&\frac{1}{p^2(1-p)}
\end{array}\]

\qpartt
$X\sim\text{Binomial}(n,p)$ where $n$ is known.\\

\apart
Let $X\sim\text{Binomial}(n,p)$ with $n$ known \& $p$ unknown. Then
\[\begin{array}{rrcl}
\text{We have}&L(p;x,n)&\propto&f(x;p,n)\\
&&=&p^x(1-p)^{n-x}\\
\implies&\ell(p;x,n)&=&x\ln p+(n-x)\ln(1-p)\\
\implies&\ell'(p;x,n)&=&\frac{x}{p}-\frac{n-x}{1-p}+c\\
&&=&\frac{x}{p}+\frac{x-n}{1-p}
\end{array}\]
\[\begin{array}{rcl}
I(p)&=&\var(\ell'(p;X,n);p)\\
&=&\var\left(\frac{X}{p}+\frac{X-n}{1-p};p\right)\\
&=&\frac{1}{p^2(1-p)^2}\var(X(1-p)+(X-n)p;p)\\
&=&\frac{1}{p^2(1-p)^2}\var(X-np;p)\\
&=&\frac{1}{p^2(1-p)^2}\var(X;p)\\
&=&\frac{1}{p^2(1-p)^2}np(1-p)\\
&=&\frac{n}{p(1-p)}
\end{array}\]

\qpartt
$X\sim\text{Normal}(\mu,\sigma^2)$ where $\sigma^2$ is known.\\

\apart
Let $X\sim\text{Normal}(\mu,\sigma^2)$ with $\sigma^2$ known \& $\mu$ unknown. Then
\[\begin{array}{rrcl}
\text{We have}&\ell(\mu;x,\sigma^2)&=&n\ln\sigma^2+\frac{1}{\sigma^2}(x-\mu)^2+c\\
\implies&\ell'(\mu;x,\sigma^2)&=&\frac{\partial}{\partial\mu}\ell(\mu;x,\sigma^2)\\
&&=&-\frac{1}{\sigma^2}2(x-\mu)\\
&I(\mu)&=&\var(\ell'(\mu;X,\sigma^2);\mu)\\
&&=&\var\left(-\frac{2}{\sigma^2}(X-\mu);\mu\right)\\
&&=&\left(\frac{2}{\sigma^2}\right)^2\var(X-\mu;\mu)\\
&&=&\frac{4}{\sigma^4}\var(X;\mu)\\
&&=&\frac{4}{\sigma^4}\sigma^2\\
&&=&\frac{4}{\sigma^2}
\end{array}\]

\qpartt
$X\sim\text{Pareto}(x_0,\theta)$ where $x_0$ is known.\\

\apart
Let $X\sim\text{Pareto}(x_0,\theta)$ with $x_0$ known \& $\theta$ unknown. Then
\[\begin{array}{rrcl}
\text{We have}&L(\theta;x,x_0)&\propto&\frac{\theta x_0^\theta}{x^{\theta+1}}\mathds{1}\{x\geq x_0\}\\
\implies&\ell(\theta;x,x_0)&=&\ln\theta+\theta\ln x_0-(\theta+1)\ln x+c\\
\implies&\ell'(\theta;x,x_0)&=&\frac{1}{\theta}+\ln x_0-\ln x\\
\implies&\ell''(\theta;x,x_0)&=&-\frac{1}{\theta^2}\\
&I(\theta)&=&-\expect(\ell''(\theta;X,x_0);\theta)\\
&&=&-\left(-\frac{1}{\theta^2}\right)\\
&&=&\frac{1}{\theta^2}
\end{array}\]

\newpage
\question
For each of the following verify that $\expect(\ell'(\theta;X);\theta)=0$.\\

\qpartt
$X\sim\text{Geometric}(p)$\\

\apart
Let $X\sim\text{Geometric}(p)$
\[\begin{array}{rrrl}
&\ell'(p;x)&=&\frac{1-x}{1-p}+\frac{1}{p}\\
\implies&\expect(\ell'(p;X);p)&=&\expect\left(\frac{1-X}{1-p}+\frac{1}{p};p\right)\\
&&=&\frac{1}{1-p}\expect(1-X)+\frac{1}{p}\\
&&=&\frac{1}{1-p}\left(1-\frac{1}{p}\right)+\frac{1}{p}\\
&&=&\frac{p-1}{p(1-p)}+\frac{1}{p}\\
&&=&\frac{p-1+(1-p)}{p(1-p)}\\
&&=&0
\end{array}\]

\qpartt
$X\sim\text{Binomial}(n,p)$ where $n$ is known.\\

\apart
Let $X\sim\text{Binomial}(n,p)$ with $n$ known.\\
\[\begin{array}{rrrl}
&\ell'(p;x,n)&=&\frac{x}{p}+\frac{x-n}{1-p}\\
\implies&\expect(\ell'(p;X,n);p)&=&\expect\left(\frac{X}{p}+\frac{X-n}{1-p};x,n\right)\\
&&=&\frac{1}{p}\expect(X)+\frac{1}{1-p}\expect(X-n)\\
&&=&\frac{1}{p}(np)+\frac{1}{1-p}(np-n)\\
&&=&n+n\frac{p-1}{1-p}\\
&&=&n-n\\
&&=&0
\end{array}\]

\qpartt
$X\sim\text{Normal}(\mu,\sigma^2)$ where $\sigma^2$ is known.\\

\apart
Let $X\sim\text{Normal}(\mu,\sigma^2)$ with $\sigma^2$ known and $\mu$ unknown.\\
\[\begin{array}{rrrl}
&\ell'(\mu;x,\sigma^2)&=&-\frac{2}{\sigma^2}(x-\mu)\\
\implies&\expect(\ell'(\mu;X,\sigma^2);\mu)&=&\expect\left(-\frac{2}{\sigma^2}(X-\mu);\mu\right)\\
&&=&-\frac{2}{\sigma^2}\expect(X-\mu;\mu)\\
&&=&-\frac{2}{\sigma^2}(\mu-\mu)\\
&&=&0
\end{array}\]

\qpartt
$X\sim\text{Pareto}(x_0,\theta)$ where $x_0$ is known.\\

\apart
Let $X\sim\text{Pareto}(x_0,\theta)$ with $x_0$ is known \& $\theta$ unknown.\\
\[\begin{array}{rrrl}
&\ell'(\theta;x,x_0)&=&\frac{1}{\theta}+\ln x_0-\ln x\\
\implies&\expect(\ell'(\theta;X,x_0);\theta)&=&\expect\left(\frac{1}{\theta}+\ln x_0-\ln X;\theta\right)\\
&&=&\frac{1}{\theta}+\ln x_0-\expect(\ln X;\theta)\\
&&=&\frac{1}{\theta}+\ln x_0-(\ln x_0+\frac{1}{\theta})\\
&&=&0
\end{array}\]
%TODO

\question
In the case of $X\sim\text{Pareto}(x_0,\theta)$ where both $x_0$ \& $\theta$ are unknown, explain why the Fisher Information Regularity Conditions are not met.\\

\ans
Let $X\sim\text{Pareto}(x_0,\theta)$ with both $x_0\ \&\ \theta$ unknown.\\
The Fisher Information Regularity Conditions require $L'(\theta,x_0;x)$ to exist $\forall\ x\in\mathcal{X}$.\\
We notice that
\[\begin{array}{rcl}
L'(\theta,x_0;x)&=&\begin{pmatrix}
\frac{\partial}{\partial \theta}L(\theta,x_0;x)&\frac{\partial}{\partial x_0}L(\theta,x_0;x)
\end{pmatrix}\\
&\propto&\begin{pmatrix}
\frac{\partial}{\partial \theta}\frac{\theta x_0^\theta}{x^{\theta+1}}\mathds{1}\{x\geq x_0\}&\frac{\partial}{\partial x_0}\frac{\theta x_0^\theta}{x^{\theta+1}}\mathds{1}\{x\geq x_0\}
\end{pmatrix}
\end{array}\]
Consider the derivative wrt $x_0$.\\
This contains an indicator function, $\mathds{1}\{x\geq x_0\}$, which depends upon the variable we are deriving wrt.\\
Since the indicator function is discontinuous at $x_0$ it is not differentiable.\\
Thus $L'(\theta,x_0;x)$ does not exist in this case, further the Fisher Information Regularity Conditions are not met in the case of $X\sim\text{Pareto}(x_0,\theta)$ with $x_0\ \&\ \theta$ unknown.$\hfill\square$

\end{document}