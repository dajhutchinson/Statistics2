\documentclass[11pt,a4paper]{article}

\usepackage[margin=1in, paperwidth=8.3in, paperheight=11.7in]{geometry}
\usepackage{amsfonts}
\usepackage{amsmath}
\usepackage{amssymb}
\usepackage{dsfont}
\usepackage{enumerate}
\usepackage{enumitem}
\usepackage{fancyhdr}
\usepackage{graphicx}
\usepackage{tikz}
\usepackage{changepage} 

\begin{document}

\pagestyle{fancy}
\setlength\parindent{0pt}
\allowdisplaybreaks

\renewcommand{\headrulewidth}{0pt}

% Cover page title
\title{Statistics 2 - Notes}
\author{Dom Hutchinson}
\date{\today}
\maketitle

% Header
\fancyhead[L]{Dom Hutchinson}
\fancyhead[C]{Statistics 2 - Notes}
\fancyhead[R]{\today}

% Counters
\newcounter{definition}[subsection]
\newcounter{example}[subsection]
\newcounter{notation}[subsection]
\newcounter{proposition}[subsection]
\newcounter{proof}[subsection]
\newcounter{remark}[subsection]
\newcounter{theorem}[subsection]

% commands
\newcommand{\dotprod}[0]{\boldsymbol{\cdot}}
\newcommand{\cosech}[0]{\mathrm{cosech}\ }
\newcommand{\cosec}[0]{\mathrm{cosec}\ }
\newcommand{\sech}[0]{\mathrm{sech}\ }
\newcommand{\prob}[0]{\mathbb{P}}
\newcommand{\nats}[0]{\mathbb{N}}
\newcommand{\cov}[0]{\mathrm{Cov}}
\newcommand{\var}[0]{\mathrm{Var}}
\newcommand{\expect}[0]{\mathbb{E}}
\newcommand{\reals}[0]{\mathbb{R}}
\newcommand{\integers}[0]{\mathbb{Z}}
\newcommand{\indicator}[0]{\mathds{1}}
\newcommand{\nb}[0]{\textit{N.B.} }
\newcommand{\ie}[0]{\textit{i.e.} }
\newcommand{\eg}[0]{\textit{e.g.} }
\newcommand{\X}[0]{\textbf{X}}
\newcommand{\x}[0]{\textbf{x}}
\newcommand{\iid}[0]{\overset{\text{iid}}{\sim}}
\newcommand{\proved}[0]{$\hfill\square$\\}

\newcommand{\definition}[1]{\stepcounter{definition} \textbf{Definition \arabic{subsection}.\arabic{definition}\ - }\textit{#1}\\}
\newcommand{\definitionn}[1]{\stepcounter{definition} \textbf{Definition \arabic{subsection}.\arabic{definition}\ - }\textit{#1}}
\newcommand{\proof}[1]{\stepcounter{proof} \textbf{Proof \arabic{subsection}.\arabic{proof}\ - }\textit{#1}\\}
\newcommand{\prooff}[1]{\stepcounter{proof} \textbf{Proof \arabic{subsection}.\arabic{proof}\ - }\textit{#1}}
\newcommand{\example}[1]{\stepcounter{example} \textbf{Example \arabic{subsection}.\arabic{example}\ - }\textit{#1}\\}
\newcommand{\examplee}[1]{\stepcounter{example} \textbf{Example \arabic{subsection}.\arabic{example}\ - }\textit{#1}}
\newcommand{\notation}[1]{\stepcounter{notation} \textbf{Notation \arabic{subsection}.\arabic{notation}\ - }\textit{#1}\\}
\newcommand{\notationn}[1]{\stepcounter{notation} \textbf{Notation \arabic{subsection}.\arabic{notation}\ - }\textit{#1}}
\newcommand{\proposition}[1]{\stepcounter{proposition} \textbf{Proposition \arabic{subsection}.\arabic{proposition}\ - }\textit{#1}\\}
\newcommand{\propositionn}[1]{\stepcounter{proposition} \textbf{Proposition \arabic{subsection}.\arabic{proposition}\ - }\textit{#1}}
\newcommand{\remark}[1]{\stepcounter{remark} \textbf{Remark \arabic{subsection}.\arabic{remark}\ - }\textit{#1}\\}
\newcommand{\remarkk}[1]{\stepcounter{remark} \textbf{Remark \arabic{subsection}.\arabic{remark}\ - }\textit{#1}}
\newcommand{\theorem}[1]{\stepcounter{theorem} \textbf{Theorem \arabic{subsection}.\arabic{theorem}\ - }\textit{#1}\\}
\newcommand{\theoremm}[1]{\stepcounter{theorem} \textbf{Theorem \arabic{subsection}.\arabic{theorem}\ - }\textit{#1}}

\tableofcontents

% Start of content
\newpage

\section{Estimation}

\subsection{Introduction}

\definition{Probabiltiy Space, $(\Omega,\mathcal{F},\prob)$}
A mathematical construct for modelling the real world. A \textit{Probabilty Space} has three elements
\begin{enumerate}[label=\roman*)]
	\item $\Omega$ - Sample space.
	\item $\mathcal{F}$ - Set of events.
	\item $\prob$ - Probability measure.
\end{enumerate}
and most fulfil the following conditions
\begin{enumerate}[label=\roman*)]
	\item $\Omega\in\mathcal{F}$;
	\item $\forall\ A\in\mathcal{F}\implies A^c\in\mathcal{F}$;
	\item $\forall\ A_0,\dots,A_n\in\mathcal{F}\implies\left(\bigcup\limits_iA_i\right)\in\mathcal{F}$;
	\item $\prob(\Omega)=1$; and,
	\item $\prob\left(\bigcup\limits_{i=1}^\infty A_i\right)=\sum_{i=1}^\infty\prob(A_i)$ for disjoint $A_1,A_2,...$ (Countable Additivity).
\end{enumerate}

\definition{Random Variable}
A function which maps an event in the sample space to a value \eg $X:\Omega\to\reals$.\\

\remark{Probability Density Function for iid Random Variable Vector}
For $\X\sim f_n(\cdot;\theta)$ where each component of $\X$ is independent and identically distribution the probability density function of $\X$ is
$$f_n(\x;\theta)=\prod\limits_{i=1}^nf(x_i;\theta)$$

\definition{Expectation}
The mean value for a random variable. For rv $X$
$$\expect(X):=\sum_{x\in\chi}xf_X(x)\quad\&\quad\expect(X):=\int_\reals xf_X(x)dx$$

\theorem{Expection of a Function}
For a function $g:\reals\to\reals$ and rv $X$ with pmf $f_X$
$$\expect(g(X)):=\sum_{g(x)\in\chi}xf_X(x)\quad\&\quad\expect(g(X)):=\int_\reals g(x)f_X(x)dx$$

\theorem{Expectation of a Linear Operator}
For rv $X$ with pmf $f_X$ \& $a,b\in\reals$
$$\expect(aX+b)=a\expect(X)+b$$

\definition{Variance}
For rv $X$
$$\var(X):=\expect\big[(X-\expect(X))^2\big]=\expect(X^2)-\expect(X)^2$$

\theorem{Variance of a Linear Operator}
For rv $X$ and $a,b\in\reals$
$$\var(aX+b)=a^2\var(X)$$

\definition{Moment of a Random Variable}
For rv $X$ the $n^{th}$ moment of $X$ is defined as $\expect(X^n)$.\\
\nb - $\expect(X^n)\neq\expect(X)^n$.\\

\definition{Covariance}
For rv $X\ \&\ Y$
$$\cov(X,Y):=\expect\big[(X-\expect(X))(Y-\expect(Y))\big]=\expect(XY)-\expect(X)\expect(Y)$$

\theorem{Properties of Covaraince}
Let $X\ \&\ Y$ be independent random varaibles
\begin{enumerate}[label=\roman*)]
	\item $\cov(X,X)=\var(X)$;
	\item $\cov(X,Y)=0$
\end{enumerate}

\theorem{Variance of two Random Variables with linear operators}
$$\var(aX+bY)=a^2\var(X)+b^2\var(Y)+2ab\cov(X,Y)$$

\theorem{Independent Random Variables}
Random variables $X_1,\dots,X_n$ are independent iff
$$\prob(X_1\leq a_1,\dots,X_n\leq a_n)=\prod_{i=1}^n\prob(X_i\leq a_i)\ \forall\ a_1,\dots,a_n\in\reals$$

%TODO frequentist v Bayesian

\subsection{The Likelihood Function}

\definition{Likelihood Function}
Define $\X\sim f_n(\cdot;\theta^*)$ for some unknown $\theta^*\in\Theta$ and let $\x$ be an observation of $\X$.\\
A \textit{Likelihood Function} is any function, $L(\cdot;\x):\Theta\to[0,\infty)$, which is proportional to the PMF/PDF of the obeserved realisation $\x$.
$$L(\theta;\x):=Cf_b(\x;\theta)\ \forall\ C>0$$
\nb Sometimes this is called the \textit{Observed} Likelihood Function since it is dependent on observed data.\\

\definition{Log-Likelihood Function}
Let $\X\sim f_n(\cdot;\theta^*)$ for some unknown $\theta^*\in\Theta$ and $\x$ be an observation of $\X$.\\
The \textit{Log-Likelihood Function} is the natural log of a \textit{Likelihood Function}
$$\ell(\theta;\x):=\ln f_n(\x;\theta)+C,\ C\in\reals$$

\theorem{Multidiensional Transforms}
Let $\X$ be a continuous random vector in $\reals^n$ with PDF $f_{\X}$; $g:\reals^n\to\reals^n$ be a continuous diferentiable bijection; and, $h:=g^{-1}$.\\
Then $\textbf{Y}=g(\X)$ is a continuous random vector and its PDF is
$$f_{\textbf{Y}}(\textbf{y})=f_{\X}(h(\textbf{y})H_h(\textbf{Y})$$
where
$$J_h:=\left|\det\left(\frac{\partial h}{\partial\textbf{y}}\right)\right|$$

\proposition{Invaraince of Likelihood Function by bijective transformation of the observations independent of $\theta$}
Let $g:\reals^n\to\reals^n$ be a bijetive transformation which is independent of $\theta$; and $\textbf{Y}:=g(\X)$.\\
Then $\textbf{Y}$ is a random variable with PDF/PMG
$$f_{\textbf{Y}}(\textbf{y};\theta)\propto f_{\X}(g^{-1}(\textbf{y});\theta)$$
Hence, if $\textbf{y}=g(\x)$ then $L_{\textbf{Y}}(\theta;\textbf{y})\propto L_{\X}(\theta;\x)$\\

\proof{Proposition 2.1}
Let $g:\reals^n\to\reals^n$ be a bijective transformation which is independent of $\theta$; $h:=g^{-1}$;  $\X,\textbf{Y}$ be a rvs st $\textbf{Y}:=g(\X)$.
\begin{enumerate}[label=\roman*)]
	\item \textit{Discrete Case} - Consider the case when $\X$ is a discrete rv. Then
	\[\begin{array}{rcl}
	f_\textbf{Y}(y;\theta)&=&\prob(\textbf{Y}=\textbf{y};\theta)\\
	&=&\prob(g^{-1}(\textbf{Y})=g^{-1}(\textbf{y});\theta)\\
	&=&\prob(h(\textbf{Y})=h(\textbf{y});\theta)\\
	&=&\prob(\X=h(\textbf{y});\theta)\\
	&=&f_\X(g^{-1}(\textbf{y});\theta)
	\end{array}\]
	\item \textit{Continuous Case} - Consider the case when $\X$ is a continuous rv.\\
	Then, by \textbf{Theorem 2.1}
	$$f_\textbf{Y}(\textbf{y};\theta)=f_\X(g^{-1}(\textbf{y});\theta)J_{g^{-1}}(\textbf{y})$$
	Since $J_{g^{-1}}$ does not depend on $\theta$ this case is solved.
\end{enumerate}
Thus in botoh cases $L_\textbf{Y}(\theta;y)=f_\textbf{Y}(y;\theta)\propto f_\X(g^{-1}(\textbf{y});\theta)=L_\X(\theta;\x)$.
\proved

\subsection{Maximum Likelihood Estimates}

\definition{Maximum Likelihood Estimate}
Let $\X\sim f_n(\cdot;\theta)$; and $\x$ be a realisation of $\X$.\\
The \textit{Maximum Likelihood Estimate} is the value $\hat{\theta}\in\Theta$ st
$$\forall\ \theta\in\Theta\ f_n(\x;\hat{\theta})\geq f_n(\x,\theta)$$
Equivalently
$$\forall\ \theta\in\Theta\ L(\hat{\theta};\x)\geq L(\theta;\x)\quad\mathrm{or}\quad\ell(\hat{\theta};\x)\geq\ell(\theta;\x)$$
\ie $\hat{\theta}(\x):=\mathrm{argmax}_\theta(L(\theta;\x)$.\\

\remark{The Maximum Likelihood Estimate may \underline{not} be unique}

\example{MLE for Uniform Distribution}
Consider $\X{\iid}U[0,\theta]$ for $\theta>0$.\\
Then
\[\begin{array}{rrcl}
&L(\theta;\x)&\propto&f_n(\x;\theta)\\
&&=&\prod\limits_{i=1}^n\dfrac{1}{\theta}\mathds{1}\{x_i\in[0,\theta]\\
&&=&\dfrac{1}{\theta^n}\prod\limits_{i=1}^n\mathds{1}\{x_i\in[0,\theta]\\
\implies&\hat{\theta}&=&\max\{x_i:x_i\in\x\}
\end{array}\]

\remark{MLE of Reparameterisation}
Define $\tau(\theta):\reals\to\reals$. Then
$$\hat{\tau}=\tau(\hat{\theta})$$
\nb We often write $\tilde{f}$ to represent the pmf when $\tau$ is taken as a parameter rateher than $\theta$. \ie $f(x;\theta)=\tilde{f}(x;\tau(\theta))$.\\

\theorem{Invariance of MLE under bijective Reparameterisation}
Let $g:\Theta\to G$ be a bijective transformation of the statisitcal parameter $\theta$.\\
Let $\X\sim f(\cdot;\theta)=\tilde{f}(\cdot;g(\theta))$ for some $\theta$, and let $\x$ be a realisation of $\X$.
\begin{center}
If $\hat{\theta}$ s an MLE of $\theta$ then $\hat{\tau}=g(\hat{\theta})$ is an MLE of $\tau$.
\end{center}

\proof{Theorem 3.1}
\textit{This is a proof by contradiction}.\\
Suppose $\exists\ \tau^*\in G st \tilde{f}(x;\tau^*)>\tilde{f}(x;\tau^*)$
We know that $\forall\ \theta\in\Theta,\ f(x;\theta)=\tilde{f}(x;g(\theta))$ and $\forall\ \tau\in G,\ f(x;g^{-1}(\tau))=\tilde{f}(x;\tau)$.\\
We deduce that
\[\begin{array}{rcl}
f(x;g^{-1}(\tau^*))&=&\tilde{f}(x;\tau^*)\\
&>&\tilde{f}(x;\hat{\tau})\text{ by assumption}\\
&=&f(x;g^{-1}(\hat{\tau}))\\
&=&f(x;\hat{\theta})
\end{array}\]
This contradicts the assumption that $\hat{\theta}$ is an maximum likelihood estimate of $\theta$.\\
\proved

\remark{Not all Reparameterisations are Bijective}
When reparameterisations $g:\reals\to\reals$ is not bijective it is helpful to consider the \textit{induced likelihood}
$$L^*(\tau;\x):=\underset{\theta\in G_\tau}{\text{max}}L(\theta;\x)\ \mathrm{where}\ G_\tau:=\{\theta:g(\theta)=\tau\}$$
Since this reduces the domain to only where $g$ is bijective.

\subsection{Determinig MLEs - The Tractable Case}

\proposition{Differentiable Likelihood in the continuous case - Multivariate}
When $L(\theta;\x$ is differentiable one can find MLEs by considering its extrema. This is done equating \& solving the cases when the gradient is zero, \ie $\nabla L(\theta;\x)=0$, and then checking whether this is a maximum or minimum point.\\
A point is a local minimum if the Hessian at the point is \textit{Negative Definite} \ie $x^TAx<0\ \forall\ x\neq\pmb{0}$.\\

\example{MLE of Normal Distribution}
Let $\X{\iid}\mathcal{N}(\mu,\sigma^2)$
\[\begin{array}{rrcl}
&L(\mu,\sigma^2;\x)&=&\prod\limits_{i=1}^n\dfrac{1}{\sqrt{2\pi\sigma^2}}e^{-\frac{(x_i-\mu)^2}{2\sigma^2}}\\
\implies&\ell(\mu,\sigma^2;\x)&=&C-\frac{n}{2}\ln(\sigma^2)-\frac{1}{2\sigma^2}\sum_{i=1}^n(x_i-\mu)^2\\
\implies&\nabla\ell(\mu,\sigma^2;\x)&=&\left(\dfrac{-1}{\sigma^2}\sum\limits_{i=1}^n(x_i-\mu),\quad-\dfrac{n}{2\sigma^2}+\dfrac{1}{2\sigma^4}\sum\limits_{i=1}^n(x_i-\mu)^2\right)\\
\text{Setting}&\dfrac{-1}{\sigma^2}\sum\limits_{i=1}^n(x_i-\mu)&=&0\\
\implies&\hat{\mu}&=&\dfrac{1}{n}\sum_{i=1}^nx_i=\bar{x}\\
\text{Setting}&-\dfrac{n}{2\sigma^2}+\dfrac{1}{2\sigma^4}\sum\limits_{i=1}^n(x_i-\mu)^2&=&0\\
\implies&\hat{\sigma}^2&=&\dfrac{1}{n}\sum\limits_{i=1}^n(x_i\hat{\mu})^2
\end{array}\]
We now want to check whether $(\hat{\mu},\hat{\sigma^2})$ is a minimum.\\
\[\begin{array}{rcl}
\nabla^2\ell(\mu,\sigma^2;\x)&=&\begin{pmatrix}
\dfrac{\partial^2\ell(\mu,\sigma^2;\x)}{\partial\mu^2}&\dfrac{\partial^2\ell(\mu,\sigma^2;\x)}{\partial\mu\partial\sigma^2}\\
\dfrac{\partial^2\ell(\mu,\sigma^2;\x)}{\partial\mu\sigma^2}&\dfrac{\partial^2\ell(\mu,\sigma^2;\x)}{\partial(\sigma^2)^2}
\end{pmatrix}\\
&=&\begin{pmatrix}
-\dfrac{n}{\hat{\sigma}^2}&0\\
0&-\dfrac{n}{2\hat{\sigma}^4}
\end{pmatrix}
\end{array}\]
Since $\begin{pmatrix}z_1&z_2\end{pmatrix}\begin{pmatrix}-a&0\\0&-b\end{pmatrix}\begin{pmatrix}z_1\\z_2\end{pmatrix}=-az_1^2-bz_2^2<0\ \forall\ a,b>0$ and we have $\frac{n}{\hat{\sigma}^2},\ \frac{n}{2\hat{\sigma}^4}>0$ then we can conclude that $\nabla^2\ell$ is negative definite.\\
Thus $\hat{\mu}=\bar{x}\ \&\ \hat{\sigma}^2=\dfrac{1}{n}\sum\limits_{i=1}^n(x_i\hat{\mu})^2$ is an MLE for the normal distribution.\\

\example{MLE for Capture-Recapture Model}
Suppose you are wanting to calculate the unknown size of a population, $n$. The Capture-Recapture Model is one technique that can be used. You tag $t\leq n$ members of the population; wait for a while; then recapture $c\leq n$ members of which $x\leq\min\{t,c\}\leq n$ are tagged.\\
With $t,c,x$ known produce a MLE for $n$.\\
We first work out the associated probability distribution for $X$, the population size. We have
\begin{enumerate}[label=\roman*)]
	\item ${t \choose x}$ ways of choosing $x$ members among the tagged ones;
	\item ${{n-t} \choose {c-x}}$ ways of choosing the remaining members among the non-tagged ones;
	\item ${n \choose c}$ ways of choosing $c$ members in a population of $n$ individuals.
\end{enumerate}
Thus
$$f_X(x;n)=\frac{{t \choose x}{{n-t} \choose {c-x}}}{{n \choose c}}$$
This means that $X\sim\text{Hypergeometric}(t,n,c)$ with $t\ \&\ c$ known.\\
Now we calculate the MLE for $X$
\[\begin{array}{rcl}
L(n;x)&=&f_X(x;n)\\
&=&\dfrac{{t \choose x}{{n-t} \choose {c-x}}}{{n \choose c}}\\
&=&\dfrac{\dfrac{t!}{x!(t-x)!}\dfrac{(n-t)!}{(c-x)!(n-t-c+x)!}}{\dfrac{n!}{c!(n-c)!}}\\
\end{array}\]
Now we consider $L(n;x)=0$ when $x>\min\{t,c\}$. We want to indetify values of $n$ for which $L(n;x)\geq L(n-1;x)$.\\
Consider $n-1\geq\min\{t,c\}\implies L(n-1;x)>0$
\[\begin{array}{rrrl}
&\text{Let }r(n)&:=&\dfrac{L(n;x)}{L(n-1;x)}\\
&&=&\dfrac{n-t}{n-t-c+x}\dfrac{n-c}{n}\\
\Rightarrow&1&\leq&r(n)\\
\Leftrightarrow&1&\leq&\dfrac{n-t}{n-t-c+x}\dfrac{n-c}{n}\\
\Leftrightarrow&n(n-t-c+x)&\leq&(n-t)(n-c)\\
\Leftrightarrow&n^2-nt-cn+xn&\leq&n^2-nt-cn+ct\\
\Leftrightarrow&xn&\leq&ct\\
\Leftrightarrow&x&\leq&\dfrac{ct}{n}
\end{array}\]
So $L(n;x)$ is increasing for $n\leq\left\lfloor\frac{ct}{x}\right\rfloor$ \& decreasing for $n>\left\lfloor\frac{ct}{x}\right\rfloor$.\\
Consequently $\hat{n}_{\text{MLE}}(x)=\left\lfloor\frac{tc}{x}\right\rfloor$

\subsection{Statistics and Estimators}

\definition{Statistic}
Given some data $\x$ a statistic is a fucntion of the data $T(\x)$.\\
\nb A statistic cannot depend on an unknown statistical parameter.\\

\definition{Estimate}
Let $\X\sim f_n(\cdot;\theta^*)$ with $\theta^*\in\Theta$ and $\x$ be a realisation of $\X$.\\
An \textit{Estimate} $\theta^*$ is a statistic $\hat{\theta}(\x)=T(\x)$ which is intended to approximate the real value of $\theta^*$.\\
\nb An \textit{Estimate} is a real value \& thus is hard to evaluate.\\

\definition{Estimator}
Let $\X\sim f_n(\cdot;\theta^*)$ with $\theta^*\in\Theta$ and $\x$ be a realisation of $\X$.\\
An \textit{Estimator} of $\theta^*$ is $\hat{\theta}$ where $\hat{\theta}(\x)$ is an \textit{estimate}.\\
\nb We call $T(\X)$ an estimator. This is a random variable.\\

\definition{Distribution of an Estimator}
Let $\X|sim f_n(\cdot;\theta^*)$ with $\theta^*\in\Theta\subseteq\reals$.\\
If $\hat{\theta}(\X)$ is a real-valued random variable, we can write its CDF as
\[\begin{array}{rcl}
F_{\hat{\theta}(\X)}(t;\theta^*)&=&\prob(\hat{\theta}(\X)\leq t;\theta^*)\\
&=&{\displaystyle \int_{\chi^n}\mathds{1}\{\hat{\theta}(\x)\leq t\}f_n(\x;\theta^*)d\x}
\end{array}\]

\remark{Estimator dependends upon true value}
The distribution of $\hat{theta}(\X)$ depends on the distribution of $\X$ which in turn depends upon the distribution of $\theta^*$.\\
Thus the distribution of an estimator depends on the true parameter of the variable it is estimating.\\

\remark{Estimator Distribution \& Sample Size}
As sample size increasesthe distribution of an estimator may converge to a more standard distribution (\eg Normal, Poisson).

\definition{Bias}
\textit{Bias} is a measure of how much an estimator deviates from the true value, on average.
\[\begin{array}{rrl}
\text{Bias}(\hat{\theta};\theta^*)&:=&\expect(\hat{\theta}(\X)-\theta^*;\theta^*)\\
&=&\expect(\hat{\theta};\theta^*)-\expect(\theta^*;\theta^*)\\
&=&\expect(\hat{\theta};\theta^*)-\theta^*
\end{array}\]

\definition{Unbiased Estimator}
An \textit{Estimator}, $\hat{\theta}$, is said to be \textit{Unbiased} if $\forall\ \theta\in\Theta,\ \text{Bias}(\hat{\theta};\theta)=0$.\\
Equivalently $\expect(\hat{\theta};\theta)=\theta$.\\

\definition{Mean Square Error}
The \textit{Mean Square Error} of an estimator is the mean of the squared error associated with rv $\hat{\theta}$.
$$MSE(\hat{\theta};\theta^*):=\expect\left[(\hat{\theta}(\X)-\theta^*)^2;\theta^2\right]$$

\proposition{Simplifcation of MSE Formula}
The MSE is a combination of variance \& bias.
\[\begin{array}{rcl}
MSE(\hat{\theta};\theta^*)&=&\expect\left[(\hat{\theta}(\X)-\theta^*)^2;\theta^2\right]\\
&=&\expect\left[\left\{\hat{\theta}-\expect(\hat{\theta};\theta^*)\right\}^2;\theta^*\right]+\left(\expect(\hat{\theta}-\theta^*;\theta^*\right)^2\\
&=&\text{Var}(\hat{\theta};\theta^*)+\text{Bias}(\hat{\theta};\theta^*)^2
\end{array}\]

\example{Sample mean as an Estimator}
Let $\X{\iid}\text{Poisson}(\lambda^*)$.\\
Suppose we are using the sample mean, $\hat{\lambda}(\x):=\frac{1}{n}\sum_{i=1}^nx_i$, as an estimate of $\lambda^*$. We first want to show this estimator is \textit{Unbiased}
\[\begin{array}{rrl}
\expect(\hat{\lambda};\lambda)&=&\expect\left(\dfrac{1}{n}\sum\limits_{i=1}^nX_i;\lambda\right)\\
&=&d\frac{1}{n}\sum\limits_{i=1}^n\expect(X_i;\lambda)\\
&=&\frac{1}{n}n\lambda\\
&=&\lambda\\
\end{array}\]
Thus $\hat{\lambda}$ is unbiased.\\
Now we consider the MSE of $\hat{\lambda}$
\[\begin{array}{rcl}
MSE(\hat{\lambda};\lambda)&=&\text{Var}(\hat{\lambda};\lambda)\\
&=&\text{Var}\left(\dfrac{1}{n}\sum\limits_{i=1}^nX_i;\lambda\right)\\
&=&\dfrac{1}{n^2}\sum\limits_{i=1}^n\text{Var}(X_i;\lambda)\\
&=&\frac{1}{n^2}n\lambda\\
&=&\frac{\lambda}{n}
\end{array}\]
This shows that as the sample size increases the MSE of $\hat{\lambda}$ converges to $0$.

\subsection{Probabilistic Convergence}

\remark{Motivation}
Here we consider the properties of a maximum likelihood estimators as the sample size increases.\\

\theorem{Markov's Inequality}
For a \textit{non-negative} random variable $X$ and a constant $a>0$
$$\prob(X\geq a)\leq\dfrac{\expect(X)}{a}$$

\proof{Markov's Inequality}
Consider continuous $X$. We have
\[\begin{array}{rrcl}
&a\prob(X\geq a)&=&a{\displaystyle\int_a^\infty f_X(x)dx}\\
&&\leq&{\displaystyle\int_a^\infty xf_X(x)dx}\\
&&\leq&{\displaystyle\int_0^\infty xf_X(x)dx}\\
&&=&\expect(X)\\
\implies&a\prob(X\geq a)&=&\expect(X)\\
\implies&\prob(X\geq a)&\leq&\dfrac{\expect(X)}{a}
\end{array}\]
\proved
% TODO discrete case

\theorem{Chebyshev's Inequality}
Let $\mu=\expect(X)$ and $\sigma^2=\text{Var}(X)$. Then
$$\forall\ a>0,\ \prob(|X-\mu|\geq a)\leq\dfrac{\sigma^2}{a^2}$$

\proof{Chebyshev's Inequality}
We have
\[\begin{array}{rcl}
\prob(|X-\mu|\geq a)&=&\prob(|X-\mu|^2\geq a^2)\\
&\leq&\dfrac{\expect\left((X-\mu)^2\right)}{a^2}\ \text{By Markov's Inequality}\\
&=&\dfrac{\sigma^2}{a^2}
\end{array}\]
\proved

\definition{Convergence in Probability}
We say the sequence of random variables $\{Z_n\}_{n\in\nats}$ converges in probability to the random variable $Z$ if
$$\forall\ \varepsilon>0,\ \lim_{n\to\infty}\prob(|Z_n-Z|>\varepsilon)=0$$
\nb This is denoted $Z_n\to_\prob Z$.\\
\nb The random variables $\{Z_n\}_{n\in\nats}$ \& $Z$ must be in the same probability space.\\

\theorem{Weak Law of Large Numbers}
If $\{X_n\}_{n\in\nats}$ are idependent \& identically distributed and $\expect(X_1)=\mu<\infty$ then
$$Z_n=\frac{1}{n}\sum_{i=1}^nX_i\to_\prob\mu$$
\nb This is an example of Convergence in Probability.\\

\definition{Convergence in Distribution}
We say the sequence of random variables $\{Z_n\}_{n\in\nats}$ converges in distribution to random variable $Z$ if
$$\forall\ z\in\text{Z where }\prob(Z\leq z)\text{ is continuous, }\lim_{n\to\infty}\prob(Z_n\leq z)=\prob(Z\leq z)$$
\nb This is denoted $Z_n\to_\mathcal{D} Z$.\\
\nb The random variables $\{Z_n\}_{n\in\nats}$ \& $Z$ need not be in the same probability space.\\

\remark{Equivalent Statements to Convergence in Distribution}
Saying that $Z_n\to_\mathcal{D} Z$ is equivalent to saying that
$$\forall\ z\in\text{Z where }F_Z(z)\text{ is continuous, }\lim_{n\to\infty}F_{Z_n}(z)=F_Z(z)$$

\theorem{Central Limit Theorem}
If $\{X_n\}_{n\in\nats}$ are idependent \& identically distributed, $\expect(X_1)=\mu<\infty$ and $\var(X_1)=\sigma^2<\infty$ then
$$\frac{\sqrt{n}}{\sigma}(Z_n-\mu)\to_\mathcal{D}Z\sim\text{Normal}(0,1)$$

\theorem{Convergence in Probability \& Distribution}
Convergence in probabilitiy $\implies$ Convergence in distribution, \textbf{but} the opposite is not necessarily true.\\

\theorem{Convergence in Probability \& Distribution to a Constant}
Convergence in distribution to a constant \textbf{and} convergence in probability to a constant are equivalent.\\

\example{}
Let $X\sim\text{Bernoulli}\left(\frac{1}{2}\right)$ and $\{X_n\}_{n\in\nats}$ be a sequence of random variables where $X_i:=(1-X)+\frac{1}{n}$.\\
We have
$$F_X(x)=\begin{cases}
0&,\ x<0\\
\frac{1}{2}&,\ x\in[0,1)\\
1&,x\geq1
\end{cases}\quad F_{X_n}(x)=\begin{cases}
0&,\ x<\frac{1}{n}\\
\frac{1}{2}&,\ x\in\left[\frac{1}{n},1+\frac{1}{n}\right)\\
1&,x\geq1+\frac{1}{n}
\end{cases}$$
Clearly $F_{X_n}(x)\to F_X(x)$ at all points at which $F_X$ is continuous (\ie $x\in\reals\backslash\{0,1\}$).\\
Thus $X_n\to_\mathcal{D}X$.\\

\theorem{Continuous Mapping Theorem}
Let $g:Z\to G$ be a \textit{continuous} function. Then
\begin{enumerate}[label=\roman*)]
	\item If $Z_n\to_\prob Z$, then $g(Z_n)\to_\prob g(Z)$;
	\item If $Z_n\to_\mathcal{D} Z$, then $g(Z_n)\to_\mathcal{D} g(Z)$
\end{enumerate}

\theorem{Slutsky's Theorem}
Let $\{Y_n\}_{n\in\nats}\ \&\ \{Z_n\}_{n\in\nats}$ be sequences of random variables, $Y$ be a random variable \& $c\in\reals\backslash0$ be a constant.\\
If $Y_n\to_\mathcal{D}Y$ and $Z_n\to_\mathcal{D}c$, then
\begin{enumerate}[label=\roman*)]
	\item $Y_n+Z_n\to_\mathcal{D}Y+c$;
	\item $Y_nZ_n\to_\mathcal{D}Yc$; and,
	\item $\frac{Y_n}{Z_n}\to_\mathcal{D}\frac{Y}{c}$.
\end{enumerate}

\definition{Convergence in Quadratic Mean}
Let $\{Z_n\}_{n\in\nats}$ be a sequence of random variables \& $Z$ be a random variable.\\
We say that $\{Z_n\}_{n\in\nats}$ \textit{Converges in Quadratic Mean} to the random variable $Z$ if
$$\lim_{n\to\infty}\expect\left[(Z_n-Z)^2\right]=0$$
\nb This is denoted $Z_n\to_{qm}Z$.\\

\theorem{If $Z_n\to_{qm}Z$ then $Z_n\to_\prob Z$}

\proof{Theorem 5.9}
Fix any $\varepsilon>0$. We have
\[\begin{array}{rcl}
\prob(|Z_n-Z|>\varepsilon)&=&\prob(|Z_n-Z|^2>\varepsilon^2)\\
&\leq&\frac{1}{\varepsilon^2}\expect\left[(Z_n-Z)^2\right]\text{ by Markov's Inequality}\\
&\to&0\text{ since $Z_n\to_{qm}Z$.}
\end{array}\]
Hence $Z_n\to_\prob Z$.\proved

\subsection{Probabilistic Convergence \& Estimators}

\definition{Consistency of a Sequence of Estimators}
A sequence of estimators, $\{\hat{\theta}_n(\cdots):\chi^n\to\Theta\}$, are said to be \textit{Consistent} if
$$\forall\ \theta\in\Theta\text{ with }\X_n\sim f_n(\cdot;\theta),\ \hat{\theta}_n(\X_n)\to_{\prob(\cdot;\theta)}\theta$$

\remarkk{Consistency of a Sequence of Estimators}
\begin{enumerate}[label=\roman*)]
	\item In numerous situations one will talk about the consistency of \textit{the} estimator, \eg for the MLE, but also for the mean, etc. This implicitly refers to the corresponding sequence of MLEs, sequence of means, etc.
	\item Note the $\prob(\cdot;\theta)$ in the limit above, and in particular the dependence on $\theta$. This is often omitted in practice, you should however not forget what the symbols actually mean.
	\item Quadratic mean / Mean Square convergence $\implies$ consistency.\\
	That is, if the MSE of the estimator converges to 0, the estimator is consistent.
\end{enumerate}

\example{Consistency of Flipping Coins}
Let $\X{\iid}\text{Bernoulli}(\theta^*)$ for some $\theta^*\in[0,1]$.\\
The maximum likelihood estimate and method of moments for $\hat{\theta}_n$ are the sample mean.\\
$${\hat{\theta}_n(X_1,\dots,X_n)=\frac{1}{n}\sum\limits_{i=1}^nX_i}$$
By the \textit{Weak Law of Large Numbers} we have that \textit{consistency} of $\{\hat{\theta}n\}$ , since $\expect(X_1)=\theta^*$.\\

\example{Crude Confidence Interval when Flipping Coins}
Let $\X{\iid}\text{Bernoulli}(\theta^*)$ for some $\theta^*\in[0,1]$ and define $\hat{\theta}_n:=\hat{\theta}_n(X_1,\dots,X_n)$.\\
We shall produce a \textit{confidence interval} for $\theta^*$.\\
	$$\expect(\hat{\theta}_n;\theta^*)=\theta^*\quad\text{and}\quad\var(\hat{\theta}_n;\theta^*)=\frac{\theta^*(1-\theta^*)}{n}$$
\[\begin{array}{rcll}
\prob\left(|\hat{\theta}_n-\theta^*|\geq\varepsilon;\theta^*\right)&\leq&\frac{\theta^*(1-\theta^*)}{n\varepsilon^2}&\text{by Chebyshev's Inequality}\\
\text{We don't know $\theta^*$, but can deduce that }\theta^*(1-\theta^*)&\leq&\frac{1}{4}\\
\implies\prob\left(|\hat{\theta}_n-\theta^*|\geq\varepsilon;\theta^*\right)&\leq&\frac{1}{4n\varepsilon^2}\\
\text{Define }\alpha&:=&\frac{1}{4n\varepsilon^2}\\
\implies\prob\left(|\hat{\theta}_n-\theta^*|\geq\frac{1}{2\sqrt{n\alpha}};\theta^*\right)&\leq&\alpha\\
\implies\prob\left(\hat{\theta}_n-\frac{1}{2\sqrt{n\alpha}}<\theta^*<\hat{\theta}_n+\frac{1}{2\sqrt{n\alpha}};\theta^*\right)&\geq&1-\alpha
\end{array}\]
This means the random interval $(\hat{\theta}_n-\frac{1}{2\sqrt{n\alpha}},\ \hat{\theta}_n+\frac{1}{2\sqrt{n\alpha}};\theta^*)$ contains $\theta^*$ with probability $1-\alpha$.\\
We can note that the interval decreases as $n$ increases, and increases as $\alpha$ decreases.
\nb $\hat{\theta}_n$ is a random variable, while $\theta^*$ is not.\\

\example{Assymptotically Exact Confidence Interval when Flipping Coins}
\textit{This is an improvement on the bound produced in }\textbf{Example 5.3}.\\
Let $\X{\iid}\text{Bernoulli}(\theta^*)$ for some $\theta^*\in[0,1]$, $W\sim\text{Normal}(0,1)$ and define $\hat{\theta}_n:=\hat{\theta}_n(X_1,\dots,X_n)$.\\
We shall show that
$$\frac{\sqrt{n}(\hat{\theta}_n-\theta^*)}{\sqrt{\hat{\theta}_n(1-\hat{\theta}_n)}}\to_\mathcal{D}W$$
We know that $\var(X_1)=\theta^*(1-\theta^*)$.\\
By the \textit{Weak Law of Large Numbers} $\hat{\theta}_n\to_\prob\theta^*$ .\\
By the \textit{Central Limit Theorem}
$$\frac{\sqrt{n}(\hat{\theta}_n-\theta^*)}{\sqrt{\hat{\theta}_n(1-\hat{\theta}_n)}}\to_\mathcal{D}W$$
Define $Y_n=\dfrac{\sqrt{n}(\hat{\theta}_n-\theta^*)}{\sqrt{\theta^*(1-\theta^*)}}$ and $Z_n=\dfrac{\sqrt{\theta^*(1-\theta^*)}}{\sqrt{\hat{\theta}_n(1-\hat{\theta}_n)}}$.\\
By the \textit{Continuous Mapping Theorem} tells us that $Z_n\to_\mathcal{D}1$ and $Z_n\to_\prob1$.\\
Hence, by \textit{Slutsky's Theorem}
$$\frac{\sqrt{n}(\hat{\theta}_n-\theta^*)}{\sqrt{\hat{\theta}_n(1-\hat{\theta}_n)}}=Y_nZ_n\to_\mathcal{D}W$$
This gives us random interval
$$\left(\hat{\theta}_n-z_{\alpha/2}\sqrt{\frac{\hat{\theta}_n(1-\hat{\theta}_n)}{n}},\hat{\theta}_n+z_{\alpha/2}\sqrt{\frac{\hat{\theta}_n(1-\hat{\theta}_n)}{n}}\right)$$
This interval captures $\theta^*$ asymptotically (in $n$) with probability $1-\alpha$.\\
\nb $z_\alpha=\Phi^{-1}(1-\alpha)$ where $\Phi$ is the cumulative denisty function of a $\text{Normal}(0,1)$.

\subsection{The Fisher Information}

\remark{Motivation}
In the next part of the content we shall show that given $\X_n{\iid}f(\cdot;\theta^*)$ then for sufficiently regular models
\begin{enumerate}[label=\roman*)]
	\item There exists a lower bound on the achievable performance of any estimate of $\theta^*$.
	\item A scaled \& centered sequence of maximum likelihood estimators $\{\hat{\theta}_n(\X_n)\}$ become asymptotically normal as $n\to\infty$.
\end{enumerate}

\remark{Measuring Performance of Estimator}
We measure the performance of an estimator $\hat{\theta}$ in terms of variance, since its mean should be $\theta^*$. Lower variance indicates better performance.\\

\definition{The Score Function}
Let $\ell(\theta;x):=\ln f(x;\theta)$.\\
The \textit{Score Function} is a measure of the sensitivity of the likelihood function wrt $\theta$
$$\ell'(\theta;x):=\frac{d}{d\theta}\ell(\theta,;x)=\frac{\frac{d}{d\theta}\ln f(x;\theta)}{\ln f(x;\theta)}=\frac{\ln L'(\theta;x)}{\ln L(\theta;x)}$$

\remark{$\theta^*$ is a turning point of $\ell(\theta;x)$}
Note that under the \textit{Fisher Information Regularity Conitions} we have that $\forall\ \theta\in\Theta$
\[\begin{array}{rcl}
\expect(\ell'(\theta;X);\theta)&=&{\displaystyle\int_S\dfrac{\frac{d}{d\theta}f(x;\theta)}{f(x;\theta)}f(x;\theta)dx}\\
&=&{\displaystyle\int_S\frac{d}{d\theta}f(x;\theta)dx}\\
&=&{\displaystyle\frac{d}{d\theta}\int_Sf(x;\theta)dx}\\
&=&\frac{d}{d\theta}(1)\\
&=&0
\end{array}\]
This shows that we expect the derivative to equal $0$ at $\theta^*$. Further, this means $\theta^*$ is a turning point of the log-likelihood function (hopefully a maximum).\\

\example{Application of \textbf{Remark 6.3}}
Let $X\sim\text{Poisson}(\theta)$. Then $f_X(x;\theta)=\frac{\theta^x}{x!}e^{-\theta}\mathds{1}\{x\in\nats\}$.
\[\begin{array}{rrcl}
\implies&\ell(\theta;x)&=&-\theta+x\ln\theta-\ln x!\\
\implies&\ell'(\theta;x)&=&-1+\frac{x}{\theta}\\
\implies&\expect(\ell'(\theta;X);\theta)&=&-1+\frac{\theta}{\theta}\\
&&=&0
\end{array}\]

\definition{Fisher Information Regularity Conditions}
Let $\Theta$ be an open interval in $\reals$ and $f(x;\theta)$ be a pmf/pdf.\\
Below are conditions which a model is required to meet in order to be considered sufficiently regular such that \textit{Fisher Information} can be drawn from it.
\begin{enumerate}[label=\roman*)]
	\item Both $L'(\theta;x)=\frac{d}{d\theta}f(x;\theta)$ and $L''(\theta;x)=\frac{d^2}{d\theta^2}f(x;\theta)$ exist for any $x\in\mathcal{X}$.
	\item $\forall\ \theta\in\Theta$ the set $S:=\{x\in\mathcal{X}:\ f(x;\theta)>0\}$ does not depend on $\theta\in\Theta$.
	\item The idenity below exists
	$$\int_S\frac{d}{d\theta}f(x;\theta)dx=\frac{d}{d\theta}\int_Sf(x;\theta)dx=0$$
\end{enumerate}

\definition{Fisher Information}
\textit{Fisher Information} is a technique for measuring the amount of information that an observable random variable $X$ carries about an unknown parameter $\theta$ upon which the probability of $X$ depends.\\
Let $X\sim f(\cdots;\theta)$. Then the \textit{Fisher Information} for any $\theta\in\Theta$ is
$$I(\theta):=\expect(\ell'(\theta;X)^2;\theta)\geq0$$
\nb This is the \textit{Expectation of the score, squared} $\equiv$ \textit{Second moment of the score}.\\

\remarkk{Fisher Information}
\begin{enumerate}[label=\roman*)]
	\item \textit{Fisher Information} is a function of the parameter, $\theta$, not the data, $X$.
	\item $I(\theta)$ can be thought of as being the average \textit{information} brought by a single observation $X$ about $\theta$, assuming $X\sim f(\cdot;\theta)$.
	\item Since $\forall\ \theta\in\Theta,\ \expect(\ell'(\theta;X);\theta)=0$ then
	$$I(\theta)=\var(\ell'(\theta;X);\theta)$$
	The variance of the score.
\end{enumerate}

\example{Fisher Information of Poisson}
Let $X\sim\text{Poisson}(\theta)$.\\
From \textbf{Example 6.1} we kown that $\ell'(\theta;x)=-1+\frac{x}{\theta}$. Then
\[\begin{array}{rcl}
I(\theta)&=&\var(\ell'(\theta;X);\theta)\\
&=&\var\left(-1+\frac{X}{\theta};\theta\right)\\
&=&\var\left(\frac{X}{\theta};\theta\right)\\
&=&\frac{1}{\theta^2}\var(X;\theta)\\
&=&\frac{1}{\theta^2}.\theta\text{ since }X\sim\text{Poisson}(\theta)\\
&=&\frac{1}{\theta}
\end{array}\]

\theorem{Alternative Expression of Fisher Information}
Let $f(x;\theta)$ be a pmf/pdf which statisfies the conditions of \textbf{Definition 6.2}. If
$$\forall\ \theta\in\Theta\quad \int_\mathcal{X}\frac{d^2}{d\theta^2}f(x;\theta)dx=\frac{d}{d\theta}\int_\mathcal{X}\frac{d}{d\theta}f(x;\theta)dx$$
Then
$$I(\theta)=-\expect\left(\frac{d^2}{d\theta^2}\ell(\theta;X);\theta\right)$$
\nb ${\displaystyle\frac{d}{d\theta}\int_\mathcal{X}\frac{d}{d\theta}f(x;\theta)dx=0}$ by the regularity conditions.\\

\proof{Theorem 6.1}
By the \textit{Quotient Rule}
\[\begin{array}{rcl}
\frac{d^2}{d\theta^2}\ell(\theta;x)&=&\dfrac{d}{d\theta}\dfrac{\frac{d}{d\theta}f(x;\theta}{f(x;\theta)}\\
&=&\dfrac{\frac{d^2}{d\theta^2}f(x;\theta)}{f(x;\theta)}-\left(\dfrac{\frac{d}{d\theta}f(x;\theta)}{f(x;\theta)}\right)^2
\end{array}\]
Consequently
\[\begin{array}{rrcl}
&\expect\left(\frac{d^2}{d\theta^2}\ell(\theta;X);\theta\right)&=&{\displaystyle\int_S\frac{\frac{d^2}{d\theta^2}f(x;\theta)}{f(x;\theta}f(x;\theta)dx-\int_S\left(\frac{\frac{d}{d\theta}f(x;\theta)}{f(x;\theta)}\right)^2f(x;\theta)dx}\\
&&=&{\displaystyle\int_S\frac{d^2}{d\theta^2}f(x;\theta)dx-\int_S\ell'(\theta;x)^2f(x;\theta)dx}\\
&&=&0-\expect(\ell'(\theta;X)^2;\theta)\\
&&=&-I(\theta)\\
\implies&I(\theta)&=&-\expect\left(\frac{d^2}{d\theta^2}\ell(\theta;X);\theta\right)
\end{array}\]
\proved

\subsection{Efficiency and The Cramer-Rao Bound}

\definition{IID Score Function}
Let $\X{\iid}f(\cdot;\theta)$ for some $\theta\in\Theta$. Then the \textit{Score Function} is
$$\ell'_n(\theta;\x):=\frac{d}{d\theta}\ell_n(\theta;\x)\text{ where }l_n(\theta;\x):=\ln f_n(\x;\theta)=\sum_{i=1}^n\ell(\theta;x_i)$$
\nb $\frac{d}{d\theta}l_n(\theta;\x)=\frac{d}{d\theta}\sum\ell(\theta;x_i)=\sum\ell'(\theta;x_i)$.\\

\definition{IID Fisher Information}
Let $\X{\iid}f(\cdot;\theta)$ for some $\theta\in\Theta$. Then the \textit{Fisher Information} is
$$I_n(\theta):=\expect(l'_n(\theta;\X)^2;\theta)=\var(l'_n(\theta;\X);\theta)$$

\theorem{Relationship between IID Fisher Information \& Fisher Information}
Consider the sitatution where $\forall\ \theta\in\Theta,\ f_n(\x;\theta)=\prod_{i=1}^nf(x_i;\theta)$. Then
$$\forall\ \theta\in\Theta,\ I_n(\theta)=nI(\theta)$$

\proof{Theorem 7.1}
Let $\X\overset{iid}{\sim}f(\cdot;\theta)$. Then
\[\begin{array}{rrcl}
&I_n(\theta)&=&\var(\ell'_n(\theta;\X);\theta)\\
&&=&\var\left(\sum\limits_{i=1}^n\ell'(\theta;X_i);\theta\right)\\
&&=&n\var\left(\sum\limits_{i=1}^n\ell'(\theta;X_1);\theta\right)\\
\implies&I_n(\theta)&=&nI(\theta)
\end{array}\]
\proved

\theorem{Cauchy-Schwarz Inequality for Expectation}
Let $X\ \&\ Y$ be real-valued random variables in the same probability space. Then
$$\expect(XY)^2\leq\expect(X^2)\expect(Y^2)$$

\proof{Theorem 7.2}
If $\expect(Y^2)=0$ then $\prob(Y=0)=1$ so $\expect(XY)=0$ and the statement holds.\\
Thus, assume $\expect(Y^2)>0$ and define $\lambda:=\dfrac{\expect(XY)}{\expect(Y^2)}$. Then
\[\begin{array}{rrcl}
&0&\leq&\expect(X-\lambda Y)^2)\\
&&=&\expect(X^2)-2\lambda\expect(XY)+\lambda^2\expect(Y^2)\\
&&=&\expect(X^2)-2\dfrac{\expect(XY)^2}{\expect(Y^2)}+\dfrac{\expect(XY)^2}{\expect(Y^2)}\\
&&=&\expect(X^2)-\dfrac{\expect(XY)^2}{\expect(Y^2)}\\
\implies&\expect(XY)^2&\leq&\expect(X^2)\expect(Y^2)
\end{array}\]
\proved

\theorem{Covaraince Inequality}
Let $X$ and $Y$ be real-valued random variables in the same probability space. Then
$$\cov(X,Y)^2\leq\var(X)\var(Y)$$

\proof{Theorem 7.3}
Let $W=X-\expect(X)$ and $Z=Y-\expect(Y)$ giving $\expect(WZ)=\cov(X,Y),\ \expect(W^2)=\var(X)$ and $\expect(Z^2)=\var(Y)$.\\
By applying the \textit{Cauchy-Schwarz inequality} we get
$$\cov(X,Y)^2=\expect(WZ)^2\leq\expect(W^2)\expect(Z^2)=\var(X)\var(Y)\Longleftrightarrow\cov(X,Y)^2\leq\var(X)\var(Y)$$

\remark{Correlation value}
The result in \textbf{Theorem 7.3} is the reason why correlation is valued in $[-1,1]$.
$$\text{Corr}(X,Y)=\dfrac{\cov(X,Y)}{\sqrt{\var(X)\var(Y)}}$$

\theorem{Cramer-Rao Inequality - Scalar Parameter}
Let $\X_n\iid f(\cdot;\theta)$ and assume the \textit{Fisher Information Regularity Conditions} hold.\\
Let $\hat{\theta}_n(\cdot)$ be an estimator of $\theta$ with expectation $m(\theta):=\expect(\hat{\theta}_n(\X_n);\theta)$ which statisfies
$$\forall\ \theta\in\Theta,\ \underbrace{\frac{d}{d\theta}\int\hat{\theta}_n(\x)f_n(\x;\theta)d\x}_{\expect(\hat{\theta}_n)}=\int\hat{\theta}_n(\x)\frac{d}{d\theta}f_n(\x;\theta)d\x$$
Then
$$\forall\ \theta\in\Theta,\quad\var(\hat{\theta}_n(\X);\theta)\geq\frac{m'(\theta)^2}{nI(\theta)}$$

\proof{Theorem 7.4}
We notice that
\[\begin{array}{rcl}
m'(\theta)&=&\frac{d}{d\theta}\expect(\hat{\theta}_n(\X_n);\theta)\\
&=&\frac{d}{d\theta}\int_{S^n}\hat{\theta}_n(\x_n)f_n(\x_n;\theta)d\x_n
\end{array}\]
The clever part of this proof is to observe that
\[\begin{array}{rcl}
\var(\hat{\theta}_n(\X_n);\theta)nI(\theta)&=&\var(\hat{\theta}_n(\X_n);\theta)\var(\ell_n(\theta;\X_n);\theta)\\
&\geq&\cov(\hat{\theta}_n(X_n),\ell'_n(\theta;\X_n);\theta)^2\text{ by Covariance Inequality}
\end{array}\]
Thus
\[\begin{array}{rrcl}
&\cov(\hat{\theta}_n(X_n),\ell'_n(\theta;\X_n);\theta)^2&=&\expect(\hat{\theta}_n(X_n)\ell_n'(\theta;\X_n);\theta)-\expect(\hat{\theta}_n(\X_n);\theta)\expect(\ell'_n(\theta;\X_n);\theta)\\
&&=&\expect(\hat{\theta}_n(X_n)\ell_n'(\theta;\X_n);\theta)-\expect(\hat{\theta}_n(\X_n);\theta)\times0\\
&&=&\expect(\hat{\theta}_n(X_n)\ell_n'(\theta;\X_n);\theta)\\
&&=&{\displaystyle\int_{S^n}\hat{\theta}_n(\x_n)\ell'_n(\theta;\x_n)f_n(\x_n;\theta)d\x_n}\\
&&=&{\displaystyle\int_{S^n}\hat{\theta}_n(\x_n)\frac{\frac{d}{d\theta}f_n(\x_n;\theta)}{f_n(\x_n;\theta)}f_n(\x_n;\theta)d\x_n}\\
&&=&{\displaystyle\int_{S^n}}\hat{\theta}_n(\x_n)\frac{d}{d\theta}f_n(\x_n;\theta)\\
&&=&\frac{d}{d\theta}{\displaystyle\int_{S^n}\hat{\theta}_n(\x_n)f_n(\x_n;\theta)d\x_n}\text{ by regularity assumption}\\
&&=&m'(\theta)\\
\implies&\var(\hat{\theta}_n(X_n);\theta)nI(\theta)&\geq&m'(\theta)^2
\end{array}\]

\proposition{Useful result from Cramer-Rao Inequality}
If $\hat{\theta}_n(\X_n)$ is an unbiased estimator (\ie $m(\theta)=\theta$) then
$$\var(\hat{\theta}_n(\X_n);\theta)=MSE(\hat{\theta}_n(\X_n);\theta)\geq\frac{1}{nI(\theta)}$$
This shows there is a lower bound on the possible performance of an estimator.\\

\definition{Efficient Estimator}
An \textit{Estimator} is said to be \textit{Efficient} when its variance is equal to the \textit{Cramer-Rao lower bound} $\forall\ \theta^*$.\\

\example{Efficient Coin Flipping}
Let $\X\iid\text{Bernoulli}(\theta)$ with $\theta\in[0,1]$, this corresponds to flipping a coin $n$ times and considering each flip the random variable $X:\{H,T\}\to\{0,1\}$ such that $X(H)=1$ and $X(T)=0$ with probability distribution such that $\prob(X=1;\theta)=\theta$ and $\prob(X=0;\theta)=1-\theta$. We consider the intuitive estimator of $\theta$
$$\hat{\theta}_n:=\hat{\theta}_n(\X_n):=\frac{1}{n}\sum_{i=1}^nX_i$$
The estimator is unbiased $\forall\ n\in\nats$ and its variance is
$$\var(\hat{\theta}_n;\theta)=\frac{\var(X_1;\theta)}{n}=\frac{\expect(X_1^2;\theta)-\expect(X_1;\theta)^2}{n}=\frac{\theta-\theta}{n}=\frac{\theta(1-\theta)}{n}$$
Now we consider the \textit{Cramer-Rao bound}
\[\begin{array}{rrcl}
\text{We find}&L(\theta;x)&=&\theta^x(1-\theta)^{1-x}\\
\implies&\ell(\theta;x)&=&x\ln\theta+(1-x)\ln(1-\theta)\\
\implies&\ell'(\theta;x)&=&\frac{x}{\theta}-\frac{1-x}{1-\theta}\\
\implies&\ell''(\theta;x)&=&-\frac{x}{\theta^2}-\frac{1-x}{(1-\theta)^2}\\
\end{array}\]
Thus we can use $I(\theta)=-\expect(\ell''(\theta;X);\theta)$
\[\begin{array}{rrcl}
\implies&I(\theta)&=&-\expect\left(-\frac{X}{\theta^2}-\frac{1-X}{(1-\theta)^2};\theta\right)\\
&&=&\expect\left(\frac{X}{\theta^2}+\frac{1-X}{(1-\theta)^2};\theta\right)\\
&&=&\frac{\theta}{\theta^2}+\frac{1-\theta}{(1-\theta)^2}\\
&&=&\frac{1}{\theta}+\frac{1}{1-\theta}\\
&&=&\frac{1}{\theta(1-\theta)}\\
&I_n(\theta)&=&nI(\theta)\text{ Since $X_1,X_2,\dots$ are iid}\\
\end{array}\]
The \textit{Cramer-Rao bound} for the variance is
$$\frac{1}{nI(\theta}=\frac{\theta(1-\theta)}{n}$$
Thus our estimator is efficient.

\subsection{Asymptotic Distribution of the Maximum Likelihood Estimator}

\theorem{}
Suppose that $\X_n\iid f(\cdot;\theta^*)$ for some $\theta^*\in\Theta$ and assume that
\begin{enumerate}[label=\roman*)]
	\item The sequence of maximum likelihood estiamtors $\{\hat{\theta}_n(\X_n)\}$ is consistent;
	\item The \textit{Fisher Information Regularity Conditions} (\textbf{Definition 6.2}) hold and $I(\theta^*)=-\expect[\ell''(\theta;X);\theta]>0$.
	\item $\exists\ C(\cdot):\mathcal{X}\to[0,\infty)$ such that $\expect(C(X_1);\theta^*)<\infty,\ \Xi\subset\Theta$ an open set containing $\theta^*$ and $\Delta(\cdot):\Xi\to[0,\infty)$ continuous at 0 st $\Delta(0)==0$, st $\forall\ \theta,\theta',x\in\Xi^2\times\mathcal{X}$.
	$$|\ell''(\theta;x)-\ell(\theta';x)|\leq C(x)\Delta(\theta-\theta')$$
\end{enumerate}
Then $\forall\ \theta^*\in\Theta$
$$\sqrt{nI(\theta^*)}(\hat{\theta}n(\X_n)-\theta^*)\to_{\mathcal{D}(\dot;\theta^*)}Z\sim\text{Normal}(0,1)$$

\theorem{}
Under the conditions of \textbf{Theorem 8.1}, with $\hat{\theta}_n:=\hat{\theta}_n(\X)$ the maximum likelihood etimator
$$\ell'_n(\hat{\theta}_n;\X)=\ell'_n(\theta^*;\X)+(\hat{\theta}_n-\theta^8)\{\ell''_n(\theta^*;\X)+R_n\}$$
where $\frac{1}{n}R_n\to_{\prob(\cdot;\theta^*)}0$.\\

\proof{Theorem 8.1}
By \textbf{Theorem 8.2} $\ell'_n(\hat{\theta}_n;\X)=\ell'_n(\theta^*;\X)+(\hat{\theta}_n-\theta^8)\{\ell''_n(\theta^*;\X)+R_n\}$ where $\frac{1}{n}R_n\to_{\prob(\cdot;\theta^*)}0$.\\
Since $\hat{\theta}_n$ is the maximum likelihood estimator \& the \textit{Fisher Information Regularity Conditions} hold, the score at $\ell'(\hat{\theta}_n;X)=0$.\\
Hence, $0=\ell''(\hat{\theta}_n;X)=\ell'_n(\theta;X)+(\hat{\theta}_n-\theta^*)\{\ell''(\theta;X)+R_n\}$.\\
Rearranging \& rescalling by $\sqrt{n}$ gives
$$\sqrt{n}(\hat{\theta}_n-\theta^*)=\frac{\frac{1}{\sqrt{n}}\ell'(\theta^*;X)}{-\frac{1}{\sqrt{n}}\{\ell''(\theta^*;X)+R_n}=:\frac{U_n}{V_n-\frac{R_n}{n}}$$
Recall that $\ell'_n(\theta^*;X)=\sum\limits_{i=1}^n\ell'(\theta;X_i)$ and $\ell''_n(\theta^*;X)=\sum\limits_{i=1}^n\ell''(\theta^*;X_i)$.\\
Since $\expect(\ell'(\theta^*;X_i);\theta^*)=0$ and $\var(\ell'(\theta^*;X_i);\theta^*)=I(\theta^*)$\\$\implies U_n\to_{\mathcal{D}(\cdot;\theta^*)}U\sim\text{Normal}(0,I(\theta^*))$ by the \textit{Central Limit Theorem}.\\
We observed that $V_n\to_{\prob(\cdot;\theta^*)}I(\theta^*)$ by the \textit{Weak Law of Large Numbers} since ${\expect(-\ell''(\theta^*;X_i);\theta^*)=I(\theta^*)}$.\\
It follows that $V_n-\frac{1}{n}R_n\to_{\prob(\cdot;\theta^*)}I(\theta^*)$ by \textit{Slutsky's Theorem}.\\
Using \textit{Slutsky's Theorem} again
$$\sqrt{n}(\hat{\theta}_n-\theta^*)=\frac{U_n}{V_n-\frac{1}{n}R_n}\to_{\mathcal{D}(\cdot;\theta^*)}\frac{\sqrt{I(\theta^*)}}{I(\theta^*)}Z\text{ where }Z\sim\text{Normal}(0,1)$$
We can rewrite this as
$$\sqrt{nI(\theta^*)}(\hat{\theta}_n-\theta^*)\to_{\mathcal{D}(\cdot;\theta^*)}Z\sim\text{Normal}(0,1)$$

\proof{Theorem 8.2}
\textit{This is an non-examinable, sketch proof of \textbf{Theorem 8.2}}.\\
By the regularity conditions and the mean alue theorem
$$\frac{\ell_n'(\theta;\x)-\ell'_n(\theta^*;\x)}{\theta-\theta^*}=\ell''_n(\tilde{\theta};\x)$$
for some $\tilde{\theta}\in(\theta,\theta^*)$. Hence, we deduce that
\[\begin{array}{rcl}
\ell'_n(\theta;\x)-\ell_n'(\theta^*;\x)&=&(\theta-\theta^*)\ell''_n(\tilde{\theta};\x)\\
&=&(\theta-\theta^*)\{\ell''_n(\theta^*;\x)+[\ell''_n(\tilde{\theta};\x)-\ell_n(\theta^*;\x)]\}\\
&=&(\theta-\theta^*)\{\ell''_n(\theta;\x)+R_n(\theta,\theta^*,\x)\}
\end{array}\]
Now we replace $\theta$ with the maximum likelihood estimator $\hat{\theta}_n:=\hat{\theta}_n(\X)$. We find
$$\ell'(\hat{\theta}_n;\X)=\ell'_n(\theta^*;\X)+(\hat{\theta}_n-\theta^*)\{\ell''_n(\theta^*;\X)+R_n(\hat{\theta}_n,\theta^*,\x\}$$
and we need to analyse $R_n$.\\
Since $\hat{\theta}_n\to_{\prob(\cdot;\theta^*)}\theta^*$ we can take $n$ large enough that $\prob(\hat{\theta}_n\in\Xi;\theta^*)$ with arbitrarily high probability.\\
On the event $\{\hat{\theta}\in\Xi\}$ and we have $\{\tilde{\theta}_n\in\Xi\}$ since $\tilde{\theta}_n\in(\hat{\theta}_n,\theta^*)$ and
\[\begin{array}{rcl}
|\frac{1}{n}R_n|&=&\frac{1}{n}|\ell''_n(\tilde{\theta}_n;\X)-\ell_n''(\theta^*;\X)|\\
&=&\dfrac{1}{n}\left|\sum\limits_{i=1}^n\ell''(\tilde{\theta}_n;X_i)-\ell''(\theta^*;X_i)\right|\\
&\leq&\dfrac{1}{n}\sum\limits_{i=1}^n\left|\ell''(\tilde{\theta}_n;X_i)-\ell''(\theta^*;X_i)\right|\\
&\leq&\Delta(\tilde{\theta}_n-\theta^*)\left\{\dfrac{1}{n}\sum\limits_{i=1}^nC(X_i)\right\}
\end{array}\]
from the smoothness condition on $\ell''$.\\
From the \textit{Weak Law of Large Numbers}
$$\frac{1}{n}\sum_{i=1}^nC(X_i)\to_{\prob(\cdot;\theta^*)}\expect(C(X_1);\theta^*)<\infty$$
and from the consistency of $\{\hat{\theta}_n\}$ and $\{\tilde{\theta}_n\}$ and continuity of $\Delta(\cdot)$ we have by the \textit{Continuous Mapping Theorem}
$$\Delta(\tilde{\theta}_n-\theta^*)\to_{\prob(\cdot;\theta^*)}0$$
Hence, $\frac{1}{n}R_n\to_{\prob(\cdot;\theta^*)}0$\proved

\definition{Asyptically Efficient}
A sequence of estimators $\{\hat{\theta}_n(\X)\}$ is \textit{Asymptotically Efficient} if either its mean-squared error converges to the \textit{Cramer-Rao Lower Bound}
$$\forall\theta\in\Theta,\ n\text{MSE}(\hat{\theta}_n(\X_n);\theta)\underset{n\to\infty}{\longrightarrow}\frac{1}{I)\theta)}$$
or $\hat{\theta}_n$ is \textit{Asumptotically Normally Distributed} in the sense of \textbf{Theorem 8.1}
$$\forall\ \theta\in\Theta,\ \sqrt{nI(\theta)}(\hat{\theta}-\theta)\to_{\mathcal{D}(\cdot;\theta)}Z$$
\nb The variance of $\frac{Z}{\sqrt{(nI(theta^*)}}$ is exactly $\frac{1}{nI(\theta)}$.\\

\theorem{}
Under the conditions of \textbf{Theorem 8.1} the maximum likelihood estimator is \textit{asymptotically efficient}.\\

\definition{Regular Statistical Model}
Any \textit{Statistical Model} which satisfies the condition of \textbf{Theorem 8.1} is a \textit{Regular Statistical Model}.\\

\remark{Why use MLE over others}
Due to the \textit{Asymptotic Efficieny} of maximum likelihood estimators it is beter to use them in \textit{Regular Statistical Models}.\\

\subsection{Confidence Sets Around the Maximum Likelihood Estimator}

\definition{Coverage of an Interval}
Let $\X\sim f_n(\cdot;\theta),\ \theta\in\Theta=\reals,\ L(\cdot):\mathcal{X}^n\to\Theta$ and $U(\cdot):\mathcal{X}^n\to\Theta$ where $\forall\ \x\in\mathcal{X}^n,\ L(\x)<U(\x)$. Then, $\forall\theta\in\Theta$ the coverage $C_\mathcal{I}(\theta)$ of the random interval $\mathcal{I}(\X):=[L(\X),U(\X)]$ at $\theta$ is
$$C_\mathcal{I}(\theta):=\prob(\theta\in[L(\X),U(\X)];\theta)=\prob(L(\X)\leq\theta\leq U(\X);\theta)$$

\remark{Coverage of an Interval in Words}
$C_\mathcal{I}(\theta)$ is the probabiltiy that the deterministic quantity $\theta$ falls into the random interval $\mathcal{I}(\X)$ under the probability distribution $\prob(\cdot;\theta)$ wher $\X\sim f_n(\cdot;\theta)$.\\

\remark{Multi-Dimensional Coverage}
We can extend \textit{Coverage of an Interval} to the multi-dimensional case by considering confidence sets and then considering the probability $\prob(\theta\in\mathcal{I}(\X);\theta)$.\\

\definition{Confidence Interval}
$\forall\ \alpha\in[0,1]$ we say that an inerval $\mathcal{I}(\X):=[L(\X),U(\X)]$ is a $1-\alpha$ confidence interval if $\forall\ \theta\in\Theta$ its coverage is at least $1-\alpha$ or more formally $\inf_{\theta\in\Theta}C_\mathcal{I}(\theta)\geq1-\alpha$.\\

\remark{Exact Confidence Interval}
If $C_\mathcal{I}(\theta)=1-\alpha\ \forall\ \theta\in\Theta$ then $\mathcal{I}$ is an exact $1-\alpha$ confidence interval.\\

\definition{Observed Confidence Interval}
For an interval $\mathcal{I}(\cdot)=[L(\cdot),U(\cdot)]$ with $L:\mathcal{X}^n\to\Theta$ and $U:\mathcal{X}^n\to\Theta$, and a realisation $\x$, the corresponding \textit{Observed Confidence Interval} is $\mathcal{I}(\x)$.\\
\nb Nothing interesting can be said about the probability that $\theta\in\mathcal{I}(\x)$ since $\theta$ and $\mathcal{I}(\x)$ are deterministic.\\

\notation{Quantile of $\text{Normal}(0,1)$}
For any $\beta\in(0,1)$ let $z_\beta\in\reals$ be such that for $Z\sim\text{Normal}(0,1)$, $1-\Phi(z_\beta)=\prob(Z>z_\beta)=\beta$.\\

\example{Confidence interval for the mean of a Normal Distribution}
Let $\X\iid\text{Normal}(\mu,\sigma^2)$ for $\theta=(\mu,\sigma^2)\in\reals\times\reals^{\geq0}$ and wher $\sigma^2$ is known.\\
Consider the estimator $\hat{\mu}_n=\hat{\mu}_n(\X)=\frac{1}{n}\sum_{i=1}^nX_i$ of $\mu$. Then we know that the following non-asymptotic result holds.\\
We have $\frac{1}{n}\sum_{i=1}^nX_i\sim\text{Normal}(\mu,\frac{\sigma^2}{n}$. Thus
$$\dfrac{\frac{1}{n}\sum_{i=1}^nX_i-\mu}{\sqrt{\sigma^2/n}}\sim\text{Normal}(0,1)$$
Then
\[\begin{array}{rcl}
\forall\ \alpha\in(0,1)&,&\prob\left(z_{1-\alpha/2}\leq\dfrac{\hat{\mu}_n(\X)-\mu}{\sqrt{\sigma^2/n}}\leq z_{\alpha/2};\mu\right)\\
&=&\prob\left(\dfrac{\hat{\mu}_n(\X)-\mu}{\sqrt{\sigma^2/n}}\leq z_{\alpha/2}\right)-\prob\left(\dfrac{\hat{\mu}_n(\X)-\mu}{\sqrt{\sigma^2/n}}\leq z_{1-\alpha/2}\right)\\
&=&\left(1-\frac{\alpha}{2}\right)-\left(1-\left(1-\frac{\alpha}{2}\right)\right)\\
&=&1-\alpha
\end{array}\]
By symmetry we notice that $z_{1-\frac{\alpha}{2}}=-z_{\alpha}{2}$.\\
By rearranging we have the equivalence of events
$$\left\{-z_{\alpha/2}\leq\frac{\hat{\mu}_n(\X)-\mu}{\sqrt{\sigma^2/n}}\leq z_{\alpha/2}\right\}=\left\{\hat{\mu}_n(\X)-z_{\alpha/2}\frac{\sigma}{\sqrt{n}}\leq\mu\leq\hat{\mu}_n(\X)+z_{\alpha/2}\frac{\sigma}{\sqrt{n}}\right\}$$
To rearrange we separate into two events \& treat then seperately
\[\begin{array}{rcl}
\left\{\dfrac{\hat{\mu}_n(\X)-\mu}{\sigma/\sqrt{n}}\leq z_{\alpha/2}\right\}&=&\left\{\dfrac{\hat{\mu}_n(\X)}{\sigma/\sqrt{n}}-z_{\alpha/2}\leq\dfrac{\mu}{\sigma/sqrt{n}}\right\}\\
&=&\left\{\mu\geq\hat{\mu}_n(\X)-\frac{\sigma}{\sqrt{n}}z_{\alpha/2}\right\}
\end{array}\]
Similarly
\[\begin{array}{rcl}
\left\{-z_{\alpha/2}\leq\dfrac{\hat{\mu}_n(X)-\mu}{\sqrt{\sigma^2/n}}\right\}&=&\left\{\dfrac{\mu}{\sigma/\sqrt{n}}\leq\dfrac{\hat{\mu}_n(X)}{\sigma/\sqrt{n}}+z_{\alpha/2}\right\}\\
&=&\left\{\mu\leq\hat{\mu}_n(X)+z_{\alpha/2}\frac{\sigma}{\sqrt{n}}\right\}
\end{array}\]
So the interval $\mathcal{I}(X)=[L(X),U(X)]$ where $L(\x)=\bar{x}-z_{\alpha/2}\frac{\sigma}{\sqrt{n}}$ and $U(\x)=\bar{x}+z_{\alpha/2}\frac{\sigma}{\sqrt{n}}$ is an $1-\alpha$ exact confidence interval.\\

\remark{Confidence Intervals with unknown $\sigma^2$}
When $\sigma^2$ is unknown we can defined $\{\hat{\sigma}^2_n\}_{n\in\nats}$ to be a consistent sequence of estimators of $\sigma^2$ (\eg the sample variance)
$$\hat{\sigma}^2_n:=\frac{1}{n-1}\sum_{i=1}^n(X_i-\hat{\mu}_n(\X))^2$$

\subsection{Asymptotic Approximation of Confidence Intervals}

\theorem{}
Assume $\X\sim f(\cdot;\theta^*)$. Let $\{\hat{\theta}_n\}_{n\in\nats}$ be a consistent sequence of estimators of $\theta^*$ and assume that $\{\hat{\theta}_n\}$ is asymptotically normal in the sense that
$$\exists\ \sigma^2>0\text{ st }\frac{\hat{\theta}_n(\X)-\theta^*}{\sqrt{\sigma^2/n}}\to_{\mathcal{D}(\cdot;\theta^*)}Z\sim\text{Normal}(0,1)$$
Then $\forall\ \alpha\in(0,1),\ \mathcal{I}_n(\X)-[L_n(\X),U_n(\X)]$ is an asymptotically exact $1-\alpha$ condifence interval, where $L_n(\x):=\hat{\theta}_n(\x)-z_{\alpha/2}\frac{\sigma}{\sqrt{n}}$ and $U_n(\x):=\hat{\theta}(\x)+z_{\alpha/2}\frac{\sigma}{\sqrt{n}}$.\\

\proof{Theorem 10.1}
Let $\{W_n\}_{n\in\nats}$ be defined by $W_n:=\frac{\hat{\theta}_n(X)-\theta^*}{\sqrt{\sigma^2/n}}$.\\
Since $W_n\to_{\mathcal{D}(\cdot;\theta^*)}Z\sim\text{Normal}(0,1)$ we have
\[\begin{array}{rcl}
\prob(-z_{\alpha/2}\leq W_n\leq z_{\alpha/2})&=&F_{W_n}(z_{\alpha/2})-F_{W_n}(-z_{\alpha/2})\\
&\underset{n\to\infty}{\longrightarrow}&\Phi(z_{\alpha/2})-\Phi(-z_{\alpha/2})\\
&=&1-\alpha
\end{array}\]
Similary to before we have the equivalence of events
$$\{-z_{\alpha/2}\leq W_n\leq z_{\alpha/2}\}=\left\{\hat{\theta}_n-z_{\alpha/2}\frac{\sigma}{\sqrt{n}}\leq\theta^*\leq\hat{\theta}_n+z_{\alpha/2}\frac{\sigma}{\sqrt{n}}\right\}$$
So $\lim_{n\to\infty}\prob\left(\hat{\theta}_n(X)-z_{\alpha/2}\frac{\sigma}{\sqrt{n}}\leq\theta^*\leq\hat{\theta}_n(X)+z_{\alpha/2}\frac{\sigma}{\sqrt{n}};\theta^*\right)=1-\alpha$

\remark{Theorem 10.1}
The confidence interval is only asymptotically exact. For finite $n$, the overage of the confidence interval will be different from $1-\alpha$ but the difference will converge to 0 as $n$ increases. In practice $\sigma^2$ may be unknown, in these cases substitute for a consistent sequence of estimators of $\sigma^2$.

\theorem{}
Assum $\X\sim f(\cdot;\theta^*)$ let $\{\hat{\theta}_n\}_{n\in\nats}$ be a consistent sequence of estimators of $\theta^*$ and assume that $\{\hat{\theta}_n\}$ is asymptotically normal in the sense that
$$\exists\ \sigma^2>0\text{ st }\frac{\hat{\theta}_n(\X)-\theta^*}{\sqrt{\sigma^2/n}}\to_{mathcal{D}(\cdot;\theta^*)}Z\sim\text{Normal}(0,1)$$
Assume also that $\{\hat{\sigma}^2_n\}_{n\in\nats}$ is a consistent sequence of estimators of $\sigma^2$. Then $\forall\ \alpha\in(0,1),\ \mathcal{I}_n(\X)=[L_n(\X),U_n(\X)]$ is an asymptotically exact $1-\alpha$ confidence interval, where $L_n(\x):=\hat{\theta}_n(\x)-z_{\alpha/2}\sqrt{\hat{\sigma}_n^2(\x)/n}$ and $U_n(\x):=\hat{\theta}_n(\x)+z_{\alpha/2}\sqrt{\hat{\sigma}_n^2(\x)/n}$.\\

\proof{Theorem 10.2}
Define $W_n:=\frac{\hat{\theta}_n-\theta^*}{\sqrt{\hat{\sigma}^2_n(X)/n}}=\frac{\hat{\theta}_n(X)-\theta^*}{\sqrt{\sigma^2/n}}-\sqrt{\frac{\sigma^2}{\hat{\sigma}^2_n(X)}}$.\\
By consistency of $\{\hat{\sigma}^2_n\}_{n\in\nats}$ and the \textit{Continuous Mapping Theorem}
$$\sqrt{\frac{\sigma^2}{\hat{\sigma}^2_n(X)}}\to_{\prob(\cdot;\theta^*)}1$$
By \textit{Slutsky's Theorem}
$$W_n\to_{\mathcal{D}(\cdot;\theta^*)}Z\sim\text{Normal}(0,1)$$
The rest of the proof is the same as for \textbf{Theorem 10.1}.\\

\remark{Theorem 10.2}
For a given $n$ the quality of the normal approximatioin will be affected by this additional approximation. One may find that for less accurate estimators of $\sigma^2$, the $n$ required for the confidence interval to have almost the right coverage will be higher.

\subsection{Estimating the Information for Maximum Likelihood Estimates}

\remark{Applying Theorem 10.2 to sequences of MLEs for regular statistical models}
When dealing with \textit{Maximum Likeihood Estimators} for regular statistical models we have that $\sigma^2=1/I(\theta^*)$ thus
$$\sqrt{nI(\theta^*)}(\hat{\theta}_n-\theta^*)\to_{\mathcal{D}(\cdot;\theta^*)}Z\sim\text{Normal}(0,1)$$
However the \textit{Fisher Information} is unknown so we consider two cases
\begin{enumerate}[label=\roman*)]
	\item When the expectation, $I(\theta^*)=-\expect(\ell''(\theta^*;X_1);\theta^*)$, can be calculated. In this case we replace $\theta^*$ with $\hat{\theta}_n$ in the equation.
	\item When the expectation \textbf{cannot} be calcualted we invoke the \textit{Weak Law of Large Numbers} and onsider the sequence of estimators, $J_n(\hat{\theta}_n):=-\frac{1}{n}\sum_{i=1}^n\ell''(\hat{\theta}_n;X_i)$.
\end{enumerate}

\theorem{Case i)}
Assume $\{\hat{\theta}_n\}$ is a sequence of \textit{Maximum Likelihood Estimators} st $\hat{\theta}_n\to_{\prob(\cdot;\theta^*)}\theta^*$ and $I$ is a continuous function of $\theta$. Then $I(\hat{\theta}_n)\to_{\prob(\cdot;\theta^*)}I(\theta^*)$.\\
\nb The proof of this follows directly from the \textit{Continous Mapping Function}.\\

\remark{Theorem 11.1}
It is only necessary for $I$ to be continuous in the neighbourhood of $\theta^*$. This is due to an extension of the \textit{Continuous Mapping Theorem} that states
\begin{center}
If $X_n\to_\prob X$ and $g$ is a function with discontinuity set $D$ then $\prob(X\in D)=0\implies (X_n)\to_\prob g(X)$.
\end{center}

\theorem{Case ii)}
Assume that $\{\hat{\theta}_n\}$ is a sequence of \textit{Maximum Likelihood Estimators} st
\begin{enumerate}[label=\roman*)]
	\item $\hat{\theta}_n\to_{\prob(\cdot;\theta^*)}\theta^*$;
	\item $I(\theta)=-\expect(\ell''(\theta;X);\theta)\ \forall\ \theta\in\Theta$
	\item $\exists\ C:\mathcal{X}\to[0,\infty)$ st $\expect(C(X_1);\theta^*)<\infty,\ \Xi\subset\Theta$ is an open set containing $\theta^*$ and $\Delta(\cdot):\Xi\to[0,\infty)$ is continuous at 0 st $\Delta(0)=0$, and st ${\forall\ \theta,\theta^*,x\in \Xi^2\times\mathcal{X}\ |\ell''(\theta;x)-\ell''(\theta';x)|\leq C(x)\Delta(\theta-\theta')}$
\end{enumerate}
Then
$$J_n(\hat{\theta}_n)\to_{\prob(\cdot;\theta^*)}I(\theta^*)$$

\proof{Theorem 11.2}
Consider the following decomposition
\[\begin{array}{rcl}
J_n(\hat{\theta})-I(\theta^*)&=&-\frac{1}{n}\sum\limits_{i=1}^n\ell''(\hat{\theta}_n;X_i)-I(\theta^*)\\
&=&T_1+T_2\\
\text{Where }T_1&=&-\frac{1}{n}\sum\limits_{i=1}^n\ell''(\hat{\theta}_n;X_i)+\frac{1}{n}\sum\limits_{i=1}^n\ell''(\theta^*;X_i)\\
\text{and }T_2&=&-\left\{\frac{1}{n}\sum\limits_{i=1}^n\ell''(\theta^*;X_i\right\}-I(\theta^*)
\end{array}\]
Now the first term can be upper bounded as follows (for sufficiently large $n$, with arbitrary large probabiltiy the second inequality holds)
\[\begin{array}{rrl}
|T_1|&=&\left|-\frac{1}{n}\sum\limits_{i=1}^n\ell''(\hat{\theta});X_i)+\frac{1}{n}\sum\limits_{i=1}^n\ell''(\theta^*;X_i)\right|\\
&\leq&\frac{1}{n}\sum\limits_{i=1}^n\left|\ell''(\hat{\theta}_n;X_i)-\ell''(\theta^*;X_i)\right|\\
&\leq&\Delta(\theta{\theta}_n-\theta^*)\frac{1}{n}\sum\limits_{i=1}^nC(X_i)
\end{array}\]
By the \textit{Weak Law of Large Numbers}
$$\frac{1}{n}\sum_{i=1}^nC(X_i)\to_{\prob(\cdot;\theta^*)}\expect(C(X_1);\theta^*)$$
by the assumed consistency of $\{\hat{\theta}_n\}_{n\in\nats}$ and continuity of $\Delta$ we have that
$$\Delta(\hat{\theta}_n-\theta^*)\to_{\prob(\cdot;\theta^*)}0$$
Consequently $T_1\overset{n\to\infty}{\longrightarrow}_{\prob(\cdot;\theta^*)}0$.\\
By the \textit{Weak Law of Large Numbers} we have
\[\begin{array}{rrrl}
&-\frac{1}{n}\sum\limits_{i=1}^n\ell''(\theta^*;X_i)&\to_{\prob(\cdot;\theta^*)}&I(\theta^*)\\
\implies&T_2=-\frac{1}{n}\sum\limits_{i=1}^n\ell''(\theta^*;X_i)-I(\theta^*)&\to_{\prob(\cdot;\theta^*)}&0
\end{array}\]
Since $T_1\overset{n\to\infty}{\longrightarrow}_{\prob(\cdot;\theta^*)}0$ and $T_2\overset{n\to\infty}{\longrightarrow}_{\prob(\cdot;\theta^*)}0$ we deduce from the earlier decomposition that
$$J_n(\hat{\theta}_n)\to_{\prob(\cdot;\theta^*)}I(\theta^*)$$
\proved

\remark{Summary}
Whenever \textbf{Theorem 8.1} holds for a sequence of \textit{Maximum Likelihood Estimators}
$$\ie\ \sqrt{nI(\theta^*)}(\hat{\theta}_n-\theta^*)\to_{\mathcal{D}(\cdot;\theta^*)}Z\sim\text{Normal}(0,1)$$
we can replace $I(\theta^*)$ with one of two options
\begin{enumerate}[label=\roman*)]
	\item $I(\hat{\theta}_n)$ whenever
	\begin{enumerate}
		\item $I(\theta)$ is continuous in a neighbourhood of $\theta^*$; and,
		\item The interval $[L(\X),U(\X)]$ with $L(\x):=\hat{\theta}_n-z_{\alpha/2}\sqrt{nI(\hat{\theta})n)}$ and $U(\x):=\hat{\theta}_n+z_{\alpha/2}\sqrt{nI(\hat{\theta})n)}$ is an asymptotically exact $1-\alpha$ confidence interval for $\theta*$.
	\end{enumerate}
	\item $J_n(\hat{\theta}_n):=-\frac{1}{n}\sum\limits_{i=1}^n\ell''(\hat{\theta}_n;X_i)$ whenever
	\begin{enumerate}
		\item The assumptions of \textbf{Theorem 11.2} hold; and,
		\item The interval $[L(\X),U(\X)]$ with $L(\x):=\hat{\theta}_n-z_{\alpha/2}\sqrt{nJ_n(\hat{\theta}_n}$ and $U(\x):=\hat{\theta}_n+z_{\alpha/2}\sqrt{nJ_n(\hat{\theta}_n}$ is an asymptotically exact $1-\alpha$ confidence interval for $\theta^*$
	\end{enumerate}
\end{enumerate}

\example{Coin Flipping}
Here the new results for this chapter are applied in order to simplfy methods used in previous examples when finding confidence intervals \& upper bounds on $\theta^*$.\\
The sequence of estimators $\hat{\theta}_n:=\frac{1}{n}\sum_{i=1}^nX_i$ is consistent by the \textit{Weak Law of Large Numbers} and the conditions for asymptotic normality hold $\forall\ \theta\in\Theta$. Hence
$$\sqrt{nI(\theta^*)}(\hat{\theta}_n-\theta^*)\to_{\mathcal{D}(\cdot;\theta^*)}Z\sim\text{Normal}(0,1)$$
We can compute the \textit{Fisher Information} $\forall\ \theta\in\Theta$. We have
\[\begin{array}{rrcl}
&\ell'(\theta(x)&=&\frac{x}{\theta}-\frac{1-x}{1-\theta}\\
\text{and }&\ell''(\theta;x)&=&-\frac{x}{\theta^2}-\frac{1-x}{(1-\theta)^2}\\
\implies&I(\theta)&=&\frac{1}{\theta}+\frac{1}{1-\theta}\\
&&=&\frac{1}{\theta(1-\theta)}
\end{array}\]
In practice $\theta^*$ is unknown so we replace $I(\theta^*)$ with $I(\hat{\theta}_n)$ to give the asymptotically exact confidence interval, $[L(\X),U(\X)]$ where
$$L(\X)=\hat{\theta}_n-z_{\alpha/2}\sqrt{\frac{\hat{\theta}_n(1-\hat{\theta}_n)}{n}}\text{ and }U(\X)=\hat{\theta}_n+z_{\alpha/2}\sqrt{\frac{\hat{\theta}_n(1-\hat{\theta}_n)}{n}}$$
If we did not know how to computer $I(\theta)$ we could instead compute
\[\begin{array}{rcl}
J_n(\hat{\theta}_n)&=&-\frac{1}{n}\sum\limits_{i=1}^n\ell''(\hat\theta_n;X_i)\\
&=&-\frac{1}{n}\sum\limits_{i=1}^n\left\{-\dfrac{X_i}{\hat\theta_n^2}-\dfrac{1-X_i}{(1-\hat\theta_n)^2}\right\}\\
&=&\frac{1}{\hat\theta_n^2}\left(\frac{1}{n}\sum\limits_{i=1}^nX_i\right)+\frac{1}{(1-\hat\theta_n)^2}\left(1-\frac{1}{n}\sum\limits_{i=1}^nX_i\right)\\
&=&\frac{\hat\theta_n}{\hat\theta_n^2}+\frac{1-\hat\theta_n}{(1-\hat\theta_n)^2}\\
&=&\frac{1}{\hat\theta_n(1-\hat\theta_n)}
\end{array}\]
In this case $J_n(\hat\theta_n)=I(\hat\theta_n)$, this is not always true.\\

\definition{Observed Fisher Information}
Let $\X\iid f(\cdot;\theta^*)$ be a vector of $n$ random variables.\\
The \textit{Observed FIsher Information} at $\theta$ is
$$nJ_n(\theta)=-\ell''(\theta;\X)=-\sum\limits_{i=1}^n\ell''(\theta;X_i)$$
\nb $\expect(J_n(\theta^*);\theta^*)=I(\theta^*)$ and that it differs from the \textit{Fisher Information} (under the \textit{Fisher Information Regularity Conditions} by not being an expectation.

\subsection{Transformations and Confidence Intervals}

\definition{Wald Approach}
The confidence intevals seen so far fit the \textit{Wald Approach}.\\
If $\X\iid f(\cdot;\theta^*)$ where $\theta^*\in\Theta\subset\reals$ then one can define a confidence interval for $\theta^*$ using the asymptotic distribution of the \textit{Maximum Likelihood Estimator}
$$L(\x)=\hat\theta_n-z_{\alpha/2}\sqrt{nI(\theta^*)}\text{ and }U(\x)=\hat\theta_n+z_{\alpha/2}\sqrt{nI(\theta^*)}$$
which ensures that as $n\to\infty,\ \prob(\theta^*\in[L(\X),U(\X)]\to1-\alpha$.\\

\proposition{Transfomed Confidence Interval - Increasing}
Let $\tau:=g(\theta)$ be a bijective, continously differentiable \& increasing function.\\
This gives a direct transformation of $[L(\x),U(\x)]$ to $[g(L(\x)),g(U(\x))]$.
$$\ie\ \{\x\in\mathcal{X}^n:L(\x)\leq\theta^*\leq U(\x)\}=\{\x\in\mathcal{X}^n:g(L(\x))\leq\tau^*\leq g(U(\x))\}$$
Consequently
\[\begin{array}{rcl}
\prob(\theta^*\in[L(\X),U(\X)];\theta^*)&=&\prob(\tau^*\in[g(L(\X)),g(U(\X))]\\
&\to&1-\alpha\text{ as }n\to\infty
\end{array}\]
\ie $[g(L(\X)),g(U(\X))]$ is an asymptotically exact $1-\alpha$ for $\tau^*$.\\

\proposition{Trasformed Confidence Interval - Decreasing}
Let $\tau:=g(\theta)$ be a bijective, continously differentiable \& decreasing function.\\
This gives a direct transformation of $[L(\X),U(\X)]$ to $[g(U(\X)),g(L(\X))]$ which is an asymptotically exact $1-\alpha$ confidence interval for $\tau^*$.\\

\remark{Deriving Reparametised Confidence Intervals}
We can obtain a reparametised \textit{Confidence Interval} by working with the reparameterised likelihood, $\tilde{f}(\x;\tau):=f(\x;g^{-1}(\tau))$. Now we can find $\tilde{L}(\x)$ and $\tilde{U}(\x)$ directly.\\

\theorem{}
Assume $X\in f(\cdot;\theta)$ for $\theta\in\Theta\subseteq\reals$ and let $\tau:=g(\theta)$ where $g$ is bijective \& continuosly diferentiable.\\
The \textit{Fisher Infromatoin} for the parameterisation $\tilde{f}(x;\tau):=f(x;g^{-1}(\tau))$ is
$$\tilde{I}(\tau)=\frac{I(\theta)}{g'(\theta)^2}$$

\proof{Theorem 12.1}
Since $\tilde{f}(x;\tau)=f(x;g^{-1}(\tau))$ the log-likelihood for $tau$ is
$$\tilde\ell(\tau;x)=\ln\tilde{f}(x;\tau)=\ln f(x;g^{-1}(\tau))$$
The score is therefore
\[\begin{array}{rcl}
\tilde\ell'(\tau;x)&=&\frac{d}{d\tau}\ln f(x;g^{-1}(\tau))\\
&=&\frac{d}{d\theta}\ln f(x;g^{-1}(\tau))\times\frac{d}{d\tau}g^{-1}(\tau)\\
&=&\ell'(g^{-1}(\tau);x)\times\dfrac{1}{g'(g^{-1}(\tau))}\\
&=&\dfrac{\ell'(\theta;x)}{g'(\theta)}
\end{array}\]
No we use the definition of \textit{Fisher Information}
\[\begin{array}{rcl}
\tilde{I}(\tau)&=&\expect(\tilde\ell'(\tau;X)^2;\tau)\\
&=&\expect\left(\dfrac{\ell'(\theta;X)^2}{g'(\theta)^2};\theta\right)\\
&=&\dfrac{1}{g'(\theta)^2}\expect\left(\ell'(\theta;X)^2;\theta\right)\\
&=&\dfrac{I(\theta)}{g'(\theta)^2}
\end{array}\]

\remark{}
As a consequence, for regular statistical models
$$\sqrt{n\tilde{I}(\tau^*)}(\hat\tau_n-\tau^*)\to_{\mathcal{D}(\cdot;\tau^*)}Z\sim\text{Normal}(0,1)$$
is equivalent to
$$\sqrt{\dfrac{nI(\theta^*)}{g'(\theta^*)^2}}(\hat\tau_n-\tau^*)\to_{\mathcal{D}(\cdot;\theta^*)}Z\sim\text{Normal}(0,1)$$
which leads to 
\[\begin{array}{rcl}
\tilde{L}(\x)&=&\hat\tau_n-z_{\alpha/2}\sqrt{\dfrac{g'(\theta^*)^2}{nI(\theta^*)}}\\
\tilde{U}(\x)&=&\hat\tau_n+z_{\alpha/2}\sqrt{\dfrac{g'(\theta^*)^2}{nI(\theta^*)}}
\end{array}\]
\nb This is not necessarily the same \textit{Confidence Interval} as obtained by transforming $[L(\x),U(\x)]$ directly.\\

\example{}
Consider $\X\iid\text{Normal}(\mu,1)$.\\
We know that the \textit{Maximum Likelihood Estiamtor} of $\mu$ is $\bar{X}\sim\text{Normal}\left(\mu,\frac{1}{n}\right)$.\\
A $1-\alpha$ \textit{Confidence Interval} for $\mu$ is
$$\left[\bar{X}-\dfrac{z_{\alpha/2}}{\sqrt{n}},\bar{X}+\dfrac{z_{\alpha/2}}{\sqrt{n}}\right]$$
Consider the parameterisation $\tau=\frac{1}{\mu}$. This corresponds to $g(x)=\frac{1}{x}$ which is bijective \& continuously differentiable except at 0, and is decreasing.\\
Hence, a $1-\alpha$ exact \textit{Confidence Interval} for $\tau$ is
$$\left[\frac{1}{\bar{X}+z_{\alpha/2}/\sqrt{n}},\frac{1}{\bar{X}-z_{\alpha/2}/\sqrt{n}}\right]$$
Consider the two ways to find an asymptotically $1-\alpha$ \textit{Exact Confidence Interval} for $\tau$. After direct calculations we find that
$$\tilde\ell''(\tau;x)=-\frac{3}{\tau^4}+\frac{2x}{\tau^3}$$
So
$$\tilde{I}(\tau):=i\expect(\tilde\ell''(\tau;X);\tau)=\frac{3}{\tau^3}-\frac{2}{\tau^4}=\frac{1}{\tau^4}$$
Noting that the \textit{Maximum Likelihood Estimator} for $\tau$ is $1/\bar{X}$ we find that
$$\sqrt{\frac{n}{\tau^4}}\left(\frac{1}{\bar{X}}-\tau\right)\to_{\mathcal{D}(\cdot;\tau)}Z\sim\text{Normal}(0,1)$$
so an asumptotically exact $1-\alpha$ \textit{Confidence Interval} is
$$\left[\frac{1}{\bar{X}}-z_{\alpha/2}\frac{\tau^2}{\sqrt{n}},\frac{1}{\bar{X}}+z_{\alpha/2}\frac{\tau^2}{\sqrt{n}}\right]$$
Alternatively, instead of working out $\tilde{I}(\tau)$ as above, we could use \textbf{Theorem 12.1} to find that
$$\tilde{I}(\tau)=\frac{I(\theta)}{g'(\theta)^2},\quad\theta=g^{-1}(\tau)=\frac{1}{\tau}$$
Since $I(\theta)=1$ and $g(\theta)=1/\theta\implies g'(\theta)=-1/\theta^2=-\tau^2$, we have
$$\tilde{I}(\tau)=\frac{1}{(-1/\theta^2)^2}=\frac{1}{(-\tau^2)^2}=\frac{1}{\tau^4}$$

\remarkk{Example 12.1}
\begin{enumerate}[label=\roman*)]
	\item The transformed \textit{Confidence Interval} is exact, which the second \textit{Confidence Interval} is not since $\sqrt{n/\tau^4}\left(\frac{1}{\bar{X}}-\tau\right)$ is not exactly normally distributed, but only asymptotically so.
	\item The transformed \textit{Confidence Interval} is not generally centred at $\hat\tau$.
	\item This serves as an example that convergence in distribution says nothing about convergence of moments. In particular, you can derive that $\frac{1}{\bar{X}}$ does not have a mean for any $\mu\in\reals$.
\end{enumerate}

\subsection{Likelihood Ratio Confidence Sets - Wilk's Approach}

\remark{Motivation}
Consider a \textit{Wald Confidence Interval} $\mathcal{I}(\theta^*)$.\\
It is possible for some $\theta\not\in\mathcal{I}(\theta^*)$ to have a greater likelihood interval than some $\theta'\in\mathcal{I}(\theta^*)$. It is possible $\exists\ \theta\in\mathcal{I}(\theta^*)$ st $L(\theta;\x)=0$.\\
\textit{Wald Confidence Intervals} are not invariant under reparameterisation.\\
These features of \textit{Wald Confidence Intervals} motivate why we may wish to consider a different type of \textit{Confidence Interval}.\\

\definition{Likelihood Ratio}
Define $\X\iid f(\cdot;\theta^*)$ for some $\theta^*\in\Theta$, let $\{\hat\theta_i\}$ be a sequence of consitent \textit{Maximum Likelihood Estimators} of $\theta^*\in\Theta$.\\
Define $\forall\ \x\in\mathcal{X}^n$ the \textit{Likelihood Ratio}
$$\Lambda_n(\x):=\frac{L(\theta^*;\x)}{L(\hat\theta_n;\x)}\in[0,1]$$

\theorem{}
Define $\X\iid f(\cdot;\theta^*)$ for some $\theta^*\in\Theta$, let $\{\hat\theta_i\}$ be a sequence of consitent \textit{Maximum Likelihood Estimators} of $\theta^*\in\Theta$ and assume that the conditions of \textbf{Theorem 8.1} hold (implying asymptotic normality). Then
$$-2\ln\Lambda_n(\X_n)\to_{\mathcal{D}(\cdot;\theta^*)}Z^2\sim\chi^2_1$$

\remark{}
We observe that
$$-2\ln\Lambda_n(\x)=-2\left(\ell(\theta^*;\x)-\ell(\hat\theta_n;\x)\right)=2\left(\ell(\hat\theta_n;\x)-\ell(\theta^*;\x)\right)$$
\ie This is twie the difference of the log-likelihoods for $\hat\theta_n$ and $\theta^*$.\\

\definition{Confidence Sets}
Define $\mathcal{X}^2_{1,\alpha}$ to be the number st $\prob(W\leq\chi^2_{1,\alpha})=1-\alpha$ for $W\sim\chi^2_1$. The \textit{Confidence Sets}
$$C(\X_n):=\left\{\theta\in\Theta:2\left[\ell(\hat\theta_n;\X_n)-\ell(\theta;\X_n)\right]\leq\chi^2_{1,\alpha}\right\}\subseteq\Theta$$
are asymptotically exact $1-\alpha$ \textit{Confidence Sets} for $\theta^*$ since
$$\prob(\theta^*\in C(\X_n;\theta^*)=\prob(-2\ln\Lambda_n(\X_n)\leq\chi^2_{1,\alpha};\theta^*)\underset{n\to\infty}{\longrightarrow}1-\alpha$$

\remark{Interpretting Confidence Sets}
$C(\x_n)$ cntains the values $\theta$ st $\ell(\theta;\x_n)$ is not too muuch less than $\ell(\hat\theta_n;\x_n)$. Hence, these confidence intervals contain those values of $\theta$ with the greatest likelihood values.\\

\remark{}
The cobserved confidence set $C(\x)$ is defined implicitly, and finding an explicit representaiton of such sets might not be easy This difficulty explains why \textit{Wald's Approach} has been historically populat, despite its shortcomings. However, with the help of a computer, it is often easy to determine $C(\x)$ numerically.

\proof{Theorem 13.1}
Consider the second order \textit{Taylor Expansion} of $\ell_n(\theta;x)=\ln f_n(x;\theta)$
$$\ell_n(\theta;x)=\ell_n(\theta_0;x)+(\theta-\theta_0)\ell_n'(\theta_0;x)+\frac{(\theta-\theta_0)^2}{2}\ell''_n(\bar\theta;x)\text{ for some }\bar\theta\in[\theta,\theta_0]$$
Rearranging we find
$$\ell_n(\theta;x)-\ell_n(\theta_0;x)=(\theta-\theta_0)\ell_n'(\theta_0;x)+\frac{(\theta-\theta_0)^2}{2}\ell_n''(\bar\theta;x)$$
Take $\theta=\theta^*$ and $\theta_0=\hat\theta_n$.\\
Since $\ell_n'(\hat\theta_n;x)-\ell_n(\hat\theta_n;x)$ then
\[\begin{array}{rrcl}
&\ln\Lambda_n(x)&=&\ell_n(\theta^*;x)-\ell_n(\hat\theta_n;x)\\
&&=&\dfrac{(\theta^*-\hat\theta_n)^2}{2}\ell_n''(\bar\theta_n;x)\text{ for some }\bar\theta_n\in[\theta^*,\hat\theta_n]\\
\implies&-2\ln\Lambda(x)&=&-(\theta^*\hat\theta_n)^2\ell_n''(\bar\theta_n;x)\\
&&=&-\left[\sqrt{nI(\theta^*)}\right]^2(\theta^*-\hat\theta_n)^2\frac{1}{nI(\theta^*)}\ell_n''(\bar\theta_n;x)
\end{array}\]
Consider the random variable $-2\ln\Lambda(X)$. Then we have
$$\sqrt{nI(\theta^*)}(\hat\theta_n(X)-\theta^*)\to_{\mathcal{D}(\cdot;\theta^*)}Z\sim\text{Normal}(0,1)$$
By the \textit{Continuous Mapping Theorem}
$$\left[\sqrt{nI(\theta^*)}\right]^2(\hat\theta_n-\theta^*)^2\to_{\mathcal{D}(\cdot;\theta^*)}Z^2$$
Since $\bar\theta_n\in[\theta^*,\hat\theta_n]$
$$-\frac{1}{n}\ell_n''(\bar\theta_n;X)\to_{\prob(\cdot;\theta^*)}I(\theta^*)$$
By \textit{Slutsky's Theorem}
$$-2\ln\Lambda_n(X)\to_{\mathcal{D}(\cdot;\theta^*)}Z^2\sim\chi^2_1$$
\proved

\remark{A Rule of Thumb}
Under the assumptions of \textbf{Theorem 13.1}, the set
$$\left\{\theta\in\Theta:\ell(\theta;\x)\geq\ell(\hat\theta_n;\x)-2\right\}$$
is an asymptotically approximate $95\%$ confidence set for $\theta^*$.\\

\proof{Remark 13.1}
We have $\chi^2_{0.05}=3.84$.\\
The result follows from the approximation $1.92\approx2$\proved

\subsection{Transformation Invariant Confidence Sets}

\remark{Motivation}
Here we investigate whether the likelihood ratio approach to determining confidence sets is invariant to transformations, in contrast to \textit{Wald's Approach}.\\
Consider the reparameterisation of the likelihood in terms of $\tau:=g(\theta)$ where $g:\Theta\to G$ is bijective. We have
$$\tilde{f}(\x;\tau):=f(\x;g^{-1}(\tau))=f(\x;\theta$$
We can now define
$$C(\x):=\left\{\theta\in\Theta:-2\left[\ell(\theta;\x)-\ell(\hat\theta_n;\x\right]\leq\chi^2_{1,\alpha}\right\}\text{ and }\tilde{C}(\x):=\left\{\theta\in\Theta:-2\left[\tilde\ell(\theta;\x)-\tilde\ell(\hat\theta_n;\x\right]\leq\chi^2_{1,\alpha}\right\}$$
We want to know whether $\theta\in C(\x)\Longleftrightarrow	g(\theta\in\tilde{C}(\x)\ \forall\ \x\in\chi^n$.\\
\ie $C(\x)\ \&\ \tilde{C}(\x)$ define the same sets up to reparameterisation.\\

\theorem{}
Let $\X\sim f(\cdot;\theta^*)$, $C$ and $\tilde{C}$ defined as in \textbf{Remark 14.1}.\\
Assume that $g:\Theta\to G$ is bijective Then
$$\forall\ \x\in\chi^n\text{ and }\theta^*\in\Theta,\ \theta\in C(\x)\Longleftrightarrow g(\theta\in\tilde{C}(\x)$$
Thus
$$\prob(\theta^*\in C(\X);\theta^*)=\prob(g(\theta^*)\in\tilde{C}(\X);\tau=g(\theta^*))$$

\proof{Theorem 14.1}
Let $\x\in\chi^n$ be arbitrary.\\
Everything rests on the observation that
$$\forall\ \theta\in\Theta,\ \ell(\theta;\x)=\ln f(\x;\theta)=\ln f(\x;g(\theta)=\tilde\ell(g(\theta;\x)$$
and similary
$$\forall\ \tau\in G,\ \tilde\ell(\tau;\x)=\ln \tilde{f}(\x;\tau)=\ln f(\x;g^{-1}(\tau))=\ell(g^{-1}(\tau);\x)$$
Note that $g(\hat\theta_n)$ is the \textit{Maximum Likelihood Estimate} of $\tau$.\\
Assume $\theta\in C(\x)$. Then
$$-2\left[\ell(\theta;\x)-\ell(\hat\theta_n;\x)\right]\leq\chi^2_{1,\alpha}$$
Thus
$$-2\left[\tilde\ell(g(\theta);\x)-\tilde\ell(g(\hat\theta_n);\x)\right]\leq\chi^2_{1,\alpha}$$
Thus $g(\theta\in\tilde{C}(\x)$.\\
So $\theta\in C(\x)\implies g(\theta)\in\tilde{C}(\x)$.\\
\\
Similarly, assume that $g(\theta)\in\tilde{C}(\x)$. Thus
$$-2\left[\ell(\theta;\x)-\ell(\hat\theta_n;\x)\right]\leq\chi^2_{1,\alpha}$$
Thus $\theta\in C(\x)$.\\
So $\theta\in C(\x)\Longleftrightarrow g(\theta)\in\tilde{X}(\x)$.\\
\\
For the last part, this correspondence implies that
$$\{\x\in\chi^n;\theta^*\in C(\x)\}=\{\x\in\chi^2:g(\theta^*)\in\tilde{C}(\x\}$$
Thus, we can conclude from the equivalnce of the events
$$\{\theta^*\in C(\X)=\{g(\theta^*)\in\tilde{C}(\X)\}$$

\section{Testing}

\subsection{Introduction to Hypothesis Tests}

\remark{Motivation}
Hypothesis testing allows us to make decisions about a parameter, rather than just estimating a rangle of values.\\

\definition{Hypothesis Testing}
\textit{Hypothesis Testing} is a process for deciding which of two competing hypotheses, $H_0$ or $H_1$, is more consistent with an observation $\x=(x_1,\dots,x_n)$ of $\X=(X_1,\dots,X_n)\sim f(\cdot;\theta)$.

\remark{Difference to Statistics 1}
In Statistics 1 we always had the null hypothesis be $H_0=\mu$. Now we consider a more general case where
\begin{enumerate}[label=\roman*)]
	\item $\X\sim f(\cdot;\theta)$ where $\theta\in\Theta$ is unknown.
	\item We have an observation $\x$ of $\X$;
	\item We have forumulate a null hypothesis concerning possible values of $\theta$ (\eg $H_0:\theta\in\Theta_0$)
	\item We have an alternative hypothsis, $H_1:\theta\in\Theta_1=\Theta\backslash\Theta_0$.
\end{enumerate}

\definition{Simple Hypothesis}
A \textit{Simple Hypothesis} is a hypothesis $H_i$ of the form $H_i:\theta=\theta_i$ where $\theta_i$ is a specified value, equivalenetly $H_i:\theta\in\Theta_i=\{\theta_i\}$.\\

\definition{Composite Hypothesis}
A \textit{Composite Hypothesis} is a hypothsis $H_i$ of the form $H_i:\theta\in\Theta_i$ where $\Theta_i$ is not a singleton. (\ie $|\Theta_i|>1$).\\

\definition{One-Sided Test}
Let $\theta$ be a scalar \& $\theta_0\in\Theta$ be a specified value.\\
A \textit{One-Sided Test} is a hypothesis test of the form
$$H_0:\theta\leq\theta_0\text{ and }H_1:\theta>\theta_0$$
or
$$H_0:\theta\geq\theta_0\text{ and }H_1:\theta<\theta_0$$

\definition{Two-Sided Test}
Let $\theta$ be a scalar \& $\theta_0\in\Theta$ be a specified value.\\
A \textit{Two-Sided Test} is a hypothesis test of the form
$$H_0:\theta=\theta_0\text{ and }H_1:\theta\neq\theta_0$$

\definition{Test Statistic}
A \textit{Test Statistic} is an operation on an observation which we use to determine the outcome of a hypothesis test. Using the distribution of specified \textit{Test Statistic} we can determine the likelihood of see a certain observation under the null-hypothesis \& thus the likelihood of the null-hypothesis being true.\\
\nb A test statistic has the signature $T:\chi^n\to\reals$.\\

\definition{Critical Value}
The \textit{Critical Value}, $c\in\reals$, is an explicit value which if the value of a test statistic $T$ exceeds it (\ie $T(\x)\geq c)$ we reject the null-hypothesis.\\

\definition{Critical Region}
The \textit{Critical Region} is the sets of observations which cause us to reject the null hypothesis.
$$R:=\{\x\in\chi^n:T(\x)\geq c\}$$
where $T$ is a \textit{Test Statistic} \& $c$ is a \textit{Critical Value}.\\
\nb $\chi^n=R\cup R^c$.\\

\subsection{Hypothesis Testing - Significance and Power}

\definition{Type I \& Type II Error}
\textit{Type I Error} occurs when $H_0$ is rejected, when in fact it is true.\\
\textit{Type II Error} occurs where $H_0$ is accepted, when in fact it is false.\\
Consider the table below\\
\begin{tabular}{|l|l|l|}
\hline
&\textbf{Retain $H_0$}&\textbf{Reject $H_0$}\\
\hline
\textbf{$H_0$ is True}&Correct&\textit{Type I Error}\\
\textbf{$H_1$ is True}&\textit{Type II Error}&Correct\\
\hline
\end{tabular}\\

\definition{Significance Level}
\textit{Significance Level} is the rate at which we allow \textit{Type I Error}s to occur
$$\alpha=\prob(\text{Type I Error})\in[0,1]$$
Typically this is the level of inprobability at which we reject the null hypothesis.\\
\nb Common \textit{Significance Levels} are $\alpha=0.05,0.01$.\\

\example{Testing the mean of a normal sample}
Suppose that $\X\iid\text{Normal}(\mu,\sigma^2)$ and we want ot test
$$H_0:\mu\leq0\text{ and }H_1:\mu>0$$
We consider the test statistic $T(\x)=\frac{1}{n}\sum_{i=1}^nx_i=\bar{x}$ with critical region
$$R:=\{\x\in\chi^n:\bar{x}\geq c\}\text{ for }c\in\reals$$
We want to find $c\in\reals$ st $\prob(X\in R;\mu\in\Theta_0)\leq\alpha\implies\prob(\bar{x}\geq c;\mu\in\Theta_0)\leq\alpha)$.\\
We know that $\bar{X}\sim\text{Normal}(\mu,sigma^2/n)$.\\
Hence $\dfrac{\bar{X}-\mu}{\sigma/\sqrt{n}}\sim\text{Normal}(0,1)$.\\
We have
$$\prob(\bar{X}\geq c;\mu)=\prob\left(\frac{\sqrt{n}(\bar{X}-\mu)}{\sigma}\geq\frac{(c-\mu)\sqrt{n}}{\sigma};\mu\right)=1-\Phi\left(\frac{\sqrt{n}(c-\mu)}{\sigma}\right)$$
We want to ensure that
\[\begin{array}{rrcl}
&\prob(\bar{X}\geq c;\mu\in\Theta_0)&\leq&\alpha\\
\Longleftrightarrow&1-\Phi\left(\frac{\sqrt{n}(c-\mu)}{\sigma}\right)&\leq&\alpha\\
\Longleftrightarrow&\frac{\sqrt{n}(c-\mu)}{\sigma}&\geq&\Phi^{-1}(1-\alpha)\\
\Longleftrightarrow&c&\geq&\mu+\frac{\sigma}{\sqrt{n}}\Phi^{-1}(1-\alpha)
\end{array}\]
Now observe that, for a fixed $c$< and considering $\mu\leq0$ and $\mu\in\Theta_0$
$$\prob(\bar{X}\geq c;\mu\in\Theta_0)\leq\prob(\bar{X}\geq c;\mu=0)$$
Thus we can esure that
$$\sup_{\mu\in\theta_0}\prob(\bar{X}\geq c;\mu)=\alpha$$
by taking $c=\frac{\sigma}{\sqrt{n}}\Phi^{-1}(1-\alpha)$.\\

\remark{Change in Critical Value}
\textit{Critical Value}, $c$, decreases as number of sample, $n$, increases.\\
\textit{Critical Value}, $c$, increases as variance, $\sigma$, increases.\\

\remark{}
\textit{Significance Level}, $\alpha$, is directly related to the phrase \textit{"statistical significance"}. \textit{Statistical Significance} relates only to the \textit{Type I Error} rate.

\subsubsection{Power}

\definition{Power Function}
Let $\X\sim f(\cdot;\theta^*)$, $T(\cdot)$ be a test statistic \& $c$ be the critical value of $T$.\\
The power function, $\pi(\cdot;T,c):\Theta\to[0,1]$, is the probabiltiy of rejecting $H_0$ when th etrue value of the parameter is $\theta\in\Theta$.
$$\pi(\theta;T,c):=\prob(\X\in R;\theta)=\prob(T(\X)\geq c;\theta)$$

\remark{}
For a given $\theta\in\Theta_1$, the probability of a \textit{Type II Error} occuring is $1-\pi(\theta;T,c)$.\\

\remarkk{}
\begin{enumerate}[label=\roman*)]
	\item THe power is non-increasing in $c$, regardless of whether $\theta\in\Theta_0$ or $\theta\in\Theta_1$.
	\item To make the probabiltiy of \textit{Type I Error} tend to 0 we should make $c$ very large so we rarely reject $H_0$.
	\item If $c$ is really large, we will rarely reject $H_0$ even if $\theta\in\Theta_1$. Thus the \textit{Power} is low and the probabilitiy of \textit{Type II Error} is high.
\end{enumerate}

\notation{}
When it is clear from context what test, $T(\cdot)$, and critical value, $c$, we are referring to then we may write $\pi(\theta)$ in place of $\pi(\theta;T,c)$.\\

\example{Testing the Mean of a Normal Sample - Continued}
Suppose that $\X\iid\text{Normal}(\mu,\sigma^2)$ and we want to test
$$H_0:\mu\leq0\text{ and }H_1:\mu>0$$
We consider the test statistic $T(\x)=\bar{x}$ with critical region $R=\{\x\in\mathcal{X}^n:\bar{x}\geq c\}$ for some $c\in\reals$.\\
The \textit{Power Function} of this test is
$$\pi(\mu;T,c)=\prob(\bar{X}\geq c;\mu)$$
We have already derived that $\dfrac{\bar{X}-\mu}{\sigma/\sqrt{n}}\sim\text{Normal}(0,1)$. Hence
\[\begin{array}{rcl}
\prob(\bar{X}\geq c;\mu)&=&\prob\left(\dfrac{\bar{X}-\mu}{\sigma/\sqrt{n}}\geq\dfrac{c-\mu}{\sigma/\sqrt{n}}\right)\\
&=&\prob\left(Z\geq\dfrac{c-\mu}{\sigma/\sqrt{n}};\mu\right)\\
&=&1-\Phi\left(\dfrac{c-\mu}{\sigma/\sqrt{n}}\right)\\
&=&\Phi\left(\dfrac{\mu-c}{\sigma/\sqrt{n}}\right)
\end{array}\]

\definition{Size of a Test}
The size of a test is the greatest possible probability of making a \textit{Type I Error}
$$\alpha=\sup_{\theta\in\Theta_0}\pi(\theta;T,c)$$
\nb It is the maximium power under the null-hypothesis.\\

\remark{}
Generally we choose a critical value $c$ so that th etest has size $\alpha$.\\

\definition{Significance Level of a Test}
A test has level $\alpha$ if its size is less than or equal to $\alpha$. The corresponding test is called a \textit{Level $\alpha$ Test}.\\

\definition{}
When $\Theta_0=\{\theta_0\}$ (\ie simple) then $\alpha=\pi(\theta_0;T,c)$ is the significan level.\\

\definition{}
When $\Theta_1=\{\theta_1\}$ (\ie simple) then $10\pi(\theta_1;T,c)$ is the probability of \textit{Type II Error}.\\

\example{Testing the mean of a normal sample - Continued}
Suppose that $\X\iid\text{Normal}(\mu,\sigma^2)$ and that we want to test
$$H_0;\mu\leq 0\text{ and }H_1:\mu>0$$
We consider the test statistic $T(\x)=\bar{x}$ with critical region $R$.\\
A test of size $\alpha$ is obtained by choosing
$$c=\frac{\sigma}{\sqrt{n}}\Phi^{-1}(1-\alpha)=\frac{\sigma}{\sqrt{n}}z_\alpha$$
So we consider the fact that $c=\frac{\sigma}{\sqrt{n}}z_\alpha$ and we obtain
$$\prob\left(\bar{X}\geq\frac{\sigma}{\sqrt{n}}z_\alpha;\mu\right)=1-\Phi\left(z_\alpha-\frac{\mu\sqrt{n}}{\sigma}\right)$$
This gives the power $\forall\ \mu\in\reals$ and we are interested in particular in it for $\mu>0$.

\subsection{Designing Tests - Neyman-Peason Approach}

\remarkk{Plan for Testing at Significance Level, $\alpha$}
\begin{enumerate}[label=\roman*)]
	\item Define a model $f(\cdot;\theta)$ for $\theta\in\Theta$
	\item Define a null hypothesis $H_0:\theta\in\Theta_0$ and an alternative hypothesis $H_1:\theta\in\Theta_1=\Theta\backslash\Theta_1$
	\item Define a test statistic $T(\x)$.
	\item Choose a critical value, $c$, st $\sup_{\theta\in\Theta_0}\prob(T(\X)\geq c;\theta)\leq\alpha$.
\end{enumerate}
\nb The value of $c$ is determined the value of $\alpha$ (which we set).

\theorem{Neyman-Pearson Lemma}
Suppose we test $H_0:\theta=\theta_0$ against $H_1:\theta=\theta_1$ and use the \textit{Likelihodd Ratio Test Statistic}
$$T_{NP}(\x):=\frac{f_n(\x;\theta_1}{f_n(\x;\theta_0)}=\frac{L(\theta_1;\x)}{L(\theta_0;\x)}$$
Let hte \textit{Critical Value}, $c_{NP}\geq0$, be st the test has size $\alpha$
$$\prob(T_{NP}\geq c_{NP};\theta_0)=\alpha$$
Then, this test is the \underline{most poweful} level $\alpha$ test.\\
\ie Among all tests with level $\alpha$, this test maximises the power function.\\

\proof{Theorem 2.1}
Cibsuder for an arbitrary level $\alpha$ test $(T,c)$, the linear combination of \textit{Type I Errors} and \textit{Type II Errors}.
$$\phi(T,c):=c_{NP}\alpha(T,c)+\beta(T,c)$$
where $\alpha(T,c)=\prob(T(\X)\geq c;\theta_0)=\prob(\text{Type I Error}$ and $\beta(T,c)=\prob(T(\X)<c;\theta_1)=1-\prob(T(\X)\geq c;\theta_1)=\prob(\text{Type II Error}$.\\
Then
\[\begin{array}{rcl}
\phi(T,c)&=&c_{NP}\alpha(T,c)+\beta(T,c)\\
&=&c_{NP}\prob(T(\X)\geq c;\theta_0)+[1-\prob(\X)\geq c;\theta_1)]\\
&=&{\displaystyle\left[c_{NP}\int\mathds{1}\{T(\x)\geq c\}f_n(\x;\theta_0)d\x\right]+\left[1-\int\mathds{1}\{T(\x)\geq c\}f_n(\x;\theta)d\x\right]}\\
&=&{\displaystyle1+\int\mathds{1}\{T(\x)\geq c\}\left[c_{NP}f_n(\x;\theta_0)-f_n(\x;\theta_1)\right]d\x}\\
&=&{\displaystyle 1+\int\mathds{1}\{T(\x)\geq c\}\left[c_{NP}-\dfrac{f_n(\x;\theta_1)}{f_n(\x;\theta_0)}\right]f_n(\x;\theta_0)d\x}\\
&=&{\displaystyle1+\int\mathds{1}\{T(\x)\geq c\}(c_{NP}-T_{NP}(\x))f_n(\x;\theta_0)d\x}
\end{array}\]
Now consider the difference
$$\phi(T,c)-\phi(T_{NP},c_{NP})=\int\big(\mathds{1}\{T(\x)\geq c\}-\mathds{1}\{T_{NP}(\x)\}\geq c_{NP}\}\big)(c_{NP}-T_{NP}(\x))f_n(\x;\theta_0)d\x$$
We observe that
$$\mathds{1}\{T_{NP}(\x)\geq c_{NP}\}=1\Longleftrightarrow c_{NP}-T_{NP}(\x)\leq0$$
and
$$\mathds{1}\{T_{NP}(\x)\geq c_{NP}\}=0\Longleftrightarrow c_{NP}-T_{NP}(\x)>0$$
Thus
$$\forall\ \x\in\mathcal{X}^n,\quad[\mathds{1}\{T(\x)\geq c\}-\mathds{1}\{T_{NP}(\x)\geq c_{NP}\}](c_{NP}-T_{NP}(\x))\geq 0$$
and hence as the integral of a non-negative function
$$\phi(T,c)-\phi(T_{NP},c_{NP})\geq0$$
We have established
\[\begin{array}{rcl}
0&\leq&\phi(T,c)-\phi(T_{NP},c_{NP})\\
&=&c_{NP}\alpha(T,c)+\beta(T,c)-c_{NP}\alpha(T_{NP},c_{NP})-\beta(T_{NP},c_{NP})\\
&=&\underbrace{c_{NP}}_{\geq0}[\alpha(T,c)-\alpha(T_{NP},c_{NP})]+\underbrace{\beta(T,c)-\beta(T_{NP},c_{NP})}_{\geq0}
\end{array}\]
Since $(T,c)$ specifides an $\alpha$ level test, we know $\alpha(T,c)\geq c$ while $(T_{NP},c_{NP})$ specifies a size $\alpha$ test so $\alpha(T_{NP},c_{NP})=\alpha$.\\
It follows that
$$\alpha(T,c)-\alpha(T_{NP},c_{NP})$$
so we have
$$\beta(T,c)-\beta(T_{NP},c_{NP})\geq0$$
which means $(T_{NP},c_{NP})$'s \textit{Type II Error} rate is no higher than $(T,c)$.\\
Since $(T,c)$ is an arbitrary $\alpha$ level test, we conclude that $(T_{NP},c_{NP})$ is the most powerful test with level $\alpha$.\proved\\

\remark{Neyman-Pearson with Non-Continuous Random Variable}
If $T(\X)$ is not a continuous random variable, then it is possible that no such $c_{NP}$ exists. In this situtation we perform an appropriate randomised test, and this will also be the most powerful size $\alpha$ test. \\
\nb The detials of this are not covered in this course.\\

\definition{Neyman-Peason Procedure}
For \textbf{Theorem 2.1} we can deduce the \textit{Neyman-Peason Procedure} for testing two simple hypotheses
\begin{enumerate}[label=\roman*)]
	\item Choose the \textit{Likelihood Ratio} as the \textit{Test Statistic}
	$$T(\x)=\frac{f_n(\x;\theta_1)}{f_n(\x;\theta_0)}=\frac{L(\theta_1;\x)}{L(\theta_0;\x)}$$
	\item Choose a critical value $c$ in order to target a particular significance level, $\alpha$, st
	$$\alpha=\pi(\theta_0)=\prob(T(\X)\geq c;\theta_0)$$
	\item Compute the \textit{Power}
	$$\pi(\theta_1,T,c)=\prob(T(\X)\geq c;\theta_1)$$
	\item Compute $T(\x)$ and report whether $T(\x)\geq c$ as well as the power $\pi(\theta_1,T,c)$ or the \textit{Type II Error} rate $1-\pi(\theta_1,T,c)$
\end{enumerate}

\remarkk{Limitations of Neyman-Pearson Approach}
\begin{enumerate}[label=\roman*)]
	\item Often just rejecting $H_0$ or retaining $H_0$ is not satisfactory, we may want more information.
	\item It is not obvious how to calibrate a likelihood ration test (\ie TO find the critical value or compute the power function).
\end{enumerate}

\subsection{Testing - p-Values, Equivalent Test Statistics and Computing the Power Function}

\remark{Motivation for p-Value}
Many studies prefer not to select in advance just one significance level $\alpha$, or they may wish to reposrt something more informative than a binary decision. IN suc cases, they can report the $p$-value associated with the observed test statistic.\\

\definition{p-Value}
Let $\X\sim f_n(\cdot;\theta^*)$ for some $\theta^*\in\Theta$.\\
The \textit{$p$-Value} for a test with test statistic $T(\x)$ is the probability of seeing a test statistic $T(\X)$ at least as extreme as $T(\x)$.
$$p(\x):=\sup_{\theta_0\in\Theta_0}\prob(\underbrace{T(\X)}_{\text{RV}}\geq \underbrace{T(\x)}_{\text{Observed}};\theta_0)$$
Equivalently, $p(\x)$ is the smallest significance level at which we would reject $H_0$.\\

\remark{p-Value Intuition}
Intuitively, \textit{$p$-value} is a measure of the eveidence against $H_0$. The smaller it is, the less likely it is that $\x$ is a realisation of $\X\sim f(\cdot;\theta_0)$, resuliting in strong evidence against $H_0$.\\
\nb A large $p$-value is not evidence in favour of $H_0$, nor is it necessarily evidence in favour of $H_1$ as $H_1$ is not involved at all when comuting the $p$-value.\\

\remark{Standard Caution}
$p(\x)$ is \textit{not} the probability that $H_0$ is true. It is the probability to observe the data we observed if $\theta_0$ is true.\\

\remark{Distribution of $p$-Value}
When using a simple null hypothesis $\Theta_0=\{\theta_0\}$ and assuming $T(\X)$ is a continuous random variable when $\X\sim f(\cdot;\theta_0)$, the distribution of $p(\X)$ is in fact uniform under the null hypothesis.\\

\example{Normal}
THe model is $\X\iid\text{Normal}(\mu,1)$ and we want to test $H_0:\mu=\mu_0<0$ against $H_1:\mu=\mu_1>0$.\\
The $p$-value for $T(\x)=\bar{x}=\frac{1}{n}\sum x_i$ is
$$p(\x):=\sup_{\mu\in\Theta_0}\prob\left(\frac{1}{n}\sum_{i=1}^nX_i\geq T(\x)=\bar{x};\mu\right)=\prob\left(\frac{1}{n}\sum_{i=1}^nX_i\geq T(\x)=\bar{x};\mu\right)$$
A very large postive value of the empirical mean leads to a small $p$-value and is an indication of how unlikely it is to have observed $\x$ if it was a realisation of $\X\iid\text{Normal}(\mu_0,1)$.\\
A large $p$-value is not an argument in favour of $H_0$, in fact it could suggest that $T(\x)$ is an unlikely realisation under $H_0$.\\
We have already calculate this kind of expression under the null hypothesis $\bar{X}\sim\text{Normal}(\mu_0,\frac{1}{n})$ so
\[\begin{array}{rcl}
\prob(\bar{X}\geq c;\mu)&=&\prob(\sqrt{n}(\bar{X}-\mu_0)\geq\sqrt{n}(c-\mu_0);\mu_0)\\
&=&\prob(Z\geq\sqrt{n}(c-\mu_0))\\
&=&1-\Phi(\sqrt{n}(c-\mu_0))
\end{array}\]
It follows that
$$p(\x)=1-\Phi(\sqrt{n}(\bar{x}-\mu_0))$$

\definition{Equivalent Statistics}
A statistic $T'(\x)$ is equivalent to $T(\x)$ if $\forall$ critical values $c\in\reals$ of $T(\cdot)$ we can find $c'\in\reals$ we can find $c'\in\reals$ st $\forall\ \x\in\mathcal{X}^n$
$$T(\x)\geq c\Longleftrightarrow T'(\x)\geq c'$$
Equivalently, $\forall\ c\in\reals$ there exist $c'\in\reals$ such that the corresponding critical regions of $T(\cdot)$ and $T'(\cdot)$ respectively conincide
$$\{\x\in\mathcal{X}^n:T(\x)\geq c\}=\{\x\in\mathcal{X}^n:T'(\x)\geq c'\}$$

\proposition{Proving Equivalence}
To verify that $T'(\x)$ is an \textit{Equivalent Statistic} to $T(\x)$ it is sufficient to factorise $T(\x)$ as
$$T(\x)=Mf(T'(\x))$$
where $M$ is independent of $\x$ and $f$ is increasing \& bijective.\\

\prooff{Proposition 4.1}
\[\begin{array}{rcl}
T(\x)\leq c&\Leftrightarrow&Mf(T'(\x))\geq c\\
&\Leftrightarrow&f(T'(\x))\geq\frac{c}{M}\\
&\Leftrightarrow&T'(\x)\leq\underbrace{f^{-1}(c/M)}_{c'}
\end{array}\]

\example{Geometric Example}
Let that $\X\iid\text{Geometric}(p)$ so that $f(x;p)=(1-p)^{x-1}p\mathds{1}\{x\in\nats\backslash\{0\}\}$.\\
Suppose that we want to test $H_0:p=p_0$ against $H_1:p=p_1$ with $p_0>p_1$.
$$T_{NP}(\x)=\frac{f_n(\x;p_1)}{f_n(\x;p_0)}=\frac{\prod_{i=1}^nf(x_i;p_1)}{\prod_{i=1}^nf(x_i;p_0)}=\prod_{i=1}^n\frac{f(\x_i;p_1)}{f(\x_i;p_0)}$$
So for $x\in X$
$$\frac{f(x;p_1)}{f(x;p_0)}=\frac{(1-p_1)^{x-1}p_1}{(1-p_0)^{x-1}p_0}=\left(\frac{1-p_1}{1-p_0}\right)^x\left(\frac{1-p_1}{1-p_0}\right)^{-1}\left(\frac{p_1}{p_0}\right)$$
So $$T_{NP}(\x)=\left(\frac{1-p_1}{1-p_0}\right)^{\sum x_i}\left(\frac{1-p_1}{1-p_0}\right)^{-n}\left(\frac{p_1}{p_0}\right)^n=\left(\frac{1-p_1}{1-p_0}\right)^{n\bar{x}}\underbrace{\left(\frac{1-p_1}{1-p_0}\right)^{-n}\left(\frac{p_1}{p_0}\right)^n}_{M}$$
Note that
$$p_0>p_1\implies 1-p_0<1-p_1\implies\frac{1-p_1}{1-p_0}>1$$
So $\left(\frac{1-p_1}{1-p_0}\right)^{n\bar{x}}$ is increasing with $\bar{x}$.\\
It follows that $T'(\x)=\bar{x}$ is an equivalent test statistic to $T_{NP}$.\\
If $\X\iid\text{Geometric}(p)$ then $n\bar{x}\sim\text{Negative-Binomical}(n,p)$.\\
Hence we can compute $c_{NP}$ or compute the power function.\\

\example{Normal Example}
The model is $\X\iid\text{Normal}(\mu,1)$ and we want to test $H_0:\mu=0$ against $H_1:\mu=1$.\\
The \textit{Neyman-Pearson Test Statistic}  is
\[\begin{array}{rcl}
T_{NP}(\x)&=&\dfrac{f_n(\x;\mu=1)}{f_n(\x;\mu=0)}\\
&=&\dfrac{\left(\frac{1}{\sqrt{2\pi}}\right)^ne^{-\frac{1}{2}\sum(x_i-1)^2}}{\left(\frac{1}{\sqrt{2\pi}}\right)^ne^{-\frac{1}{2}\sum(x_i-0)^2}}\\
&=&e^{-\frac{1}{2}\left(\sum x_i^2-2\sum x_i+n-\sum x_i^2\right)}\\
&=&e^{-\frac{1}{2}(-2n\bar{x}+n)}\\
&=&\underbrace{e^{-\frac{n}{2}}}_Me^{n\bar{x}}
\end{array}\]
Since $T_{NP}$ is increasing in terms of $\bar{x}$, $\bar{x}$ is an equivalet test statistic to $T_{NP}$.\\
To relate $T_{NP}\geq c_{NP}$ with $T(\x)=\bar{x}\geq c$.\\
We have
\[\begin{array}{rclcrcl}
T_{NP}(\x)&\geq&c_{NP}&\Longleftrightarrow&e^{n\bar{x}-\frac{n}{2}}&\geq&c_{NP}\\
&&&\Longleftrightarrow&n\bar{x}-\frac{n}{2}&\geq&\ln c_{NP}\\
&&&\Longleftrightarrow&\bar{x}-\frac{1}{2}&\geq&\frac{1}{n}\ln c_{NP}\\
&&&\Longleftrightarrow&\bar{x}&\geq&\underbrace{\frac{1}{2}+\frac{1}{n}\ln c_{NP}}_c
\end{array}\]
So $c_{NP}=e^{n(c-\frac{1}{n})}$.\\
Now we can compute the power function.\\
We know that $\bar{X}\sim\text{Normal}(\mu,\frac{1}{n})$ for $\alpha\in(0,1)$.\\
We find $c$ by solving
\[\begin{array}{rrcl}
&\pi(\mu_0;T,c)&=&\alpha\\
\implies&\prob(\bar{X}\geq c;\mu_0)&=&\alpha\\
\implies&\prob\left(Z\geq\frac{c-0}{1/\sqrt{n}}\right)&=&\alpha\\
\implies&1-\Phi(c\sqrt{n})&=&\alpha\\
\implies&\Phi(c\sqrt{n})&=&1-\alpha\\
\implies&c\sqrt{n}&=&\Phi^{-1}(1-\alpha)\\
\implies&c&=&\dfrac{\Phi^{-1}(1-\alpha)}{\sqrt{n}}\\
&&=&\dfrac{z_{\alpha}}{\sqrt{n}}
\end{array}\]
Hence $c_{NP}=e^{n\left(\frac{z_\alpha}{\sqrt{n}}-\frac{1}{2}\right)}$
We can also compute \textit{Type II Error} probability
\[\begin{array}{rcl}
1-\pi(1)&=&\prob(\bar{X}<c;\mu=1)\\
&=&\prob\left(Z<\frac{c-1}{1/\sqrt{n}}\right)\\
&=&\Phi(\sqrt{n}c-\sqrt{n})\\
&=&\Phi(z_\alpha-\sqrt{n})\overset{n\to\infty}{\longrightarrow}0
\end{array}\]

\subsection{Uniformly Most Powerful Tests}

\definition{Uniformly Most Powerful Test}
Consider a test involving composite hypothese $H_0:\theta\leq\theta_0$ against $H_1:\theta>\theta_0$.\\
A \textit{Uniformly Most Powerful Test} is a test $(T,c)$ which has the largest power $\pi(\theta;T,c)$ among all possible tests, uniformly in $\theta\in\Theta_1$. That is a $(T,c)$ st $\forall\theta\in\Theta_1$ and any test statistic $(T',c')$
$$\pi(\theta;T,c)\geq\pi(\theta;T',c')$$

\remark{}
The \textit{Type II Error Rate} depends on a specific value of $\theta\in\Theta_1$. Typcally, the \textit{Type II Error Rate} is close to $1-\alpha$ for values of $\theta\in\Theta_1$ \textit{"very close to being in "} $\Theta_0$. \ie $\pi(\theta;T,c)\approx\alpha$ for $\theta=\theta_0+\varepsilon$ for $\varepsilon$ very small.\\

\theorem{}
Let $\Theta_1=\{\theta:\theta>\theta_0\}$ for some $\theta_0\in\Theta$.\\
Assume that for the simple hypothese
$$H_0':\theta=\theta_1\quad\text{against}\quad H_1':\theta=\theta_2$$
The \textit{Neyman-Pearson Test Statistic}
$$T_{NP}(\x)=\frac{f_n(\x;\theta_2)}{f_n(\x;\theta_1)}$$
is equivalent to the same test statistic $T(\x)$ for any $\theta_1<\theta_2$ and $T(\x)$ does not depend on $\theta_1$ or $\theta_2$.\\
Then $T(\x)$ is the uniformly most powerful test statistic for
$$H_0:\theta\leq\theta_0\quad\text{against}\quad H_1:\theta>\theta_0$$
and the associated $p$-value is
$$p(\x)=\prob(T(\X)\geq T(\x);\theta_0)$$

\example{Poisson}
Let $\X\iid\text{Poisson}(\lambda)$ for some $\lambda>0$ and we want to test $H_0:\lambda\leq\lambda_0$ against $H_1:\lambda>\lambda_0$.\\
Compute the $p$-value associated with this test.\\
Consider $H_0:\lambda=\lambda_0$ \& $H_1:\lambda=\lambda_1$ where $\lambda_1>\lambda_0$.\\
$$T_{NP}(\x)=\prod_{i=1}^n\left(\dfrac{e^{-\lambda_1}\lambda_1^{x_i}(x_i!)^{-1}}{e^{-\lambda_0}\lambda_0^{x_i}(x_i!)^{-1}}\right)=e^{-n(\lambda_1-\lambda_0)}\left(\frac{\lambda_1}{\lambda_0}\right)^{\sum x_i}$$
Since $\lambda_1>\lambda_0$ we have $T_{NP}(\x)$ is an increasing function in terms of $\sum x_i$.\\
So $S_n:=\sum x_i$ is an equivalent test statistic and does not depend on $\lambda_0$ or $\lambda_1$.\\
Hence, $S_n$ is a \textit{Uniformly Most Powerful Test}.
To find the $p$-value
$$p(\x)=\prob(S_n(\X)\geq S_n(\x);\lambda_0)$$
We use $\lambda_0$ in this scenario since it is the vlaue which is most likely to produce extreme values, in general, for $T(\X)$.\\
We know that $S_n(\X)=\sum X_i\sim\text{Poisson}(n\lambda)$. So
$$p(\x)=\sum_{k\geq S_n(\x)}e^{-\lambda_0n}(n\lambda_0)^k\frac{1}{k!}$$
\textit{Alternatively}\\
If $n$ is alrge $S_n(\X)\simeq\text{Normal}(n\lambda,n\lambda)$.\\
So
$$p(\x)\simeq\prob\left(Z\geq\dfrac{S_n(\x)-n\lambda_0}{\sqrt{n\lambda_0}}\right)=1-\Phi\left(\dfrac{S_n(\x)-n\lambda_0}{\sqrt{n\lambda_0}}\right)$$
where $Z\sim\text{Normal}(0,1)$.\\

\example{Geometric}
Let $\X\iid\text{Geometric}(p)$ we have already shown that for two simple hypotheses $\bar{X}$ is equivalent to the likelihood ratio test statistic when $p_0>p_1$.\\
It follows that $T(\x)=S_n(\x)$ is equivalent.\\
We have noticed that $S_n(\X)\sim\text{NegBinomial}(n,p)$.\\
We shall compute the $p$-value assocaited to hypotheses $H_0:p\leq p_0$ against $H_1:p>p_0$.\\
From \textbf{Example 2.4.2} we have that for $H_0':p=p_0$ against $H_1':p=p_1$
$$T_{NP}(\x)=\left(\frac{p_1}{p_0}\right)^n\left(\frac{1-p_1}{1-p_0}\right)^{-n}\left(\frac{1-p_1}{1-p_0}\right)^{n\bar{x}}$$
For $P_1>p_0$ we see that $T_{NP}$ is a decreasing function (since the last two terms are $<0$) in terms of $\sum X_i=:S_n$.\\
Hence it is increasing in terms of $T(\x):=-S_n(\x)$.\\
Since $S_n$ is independent of $p_0$ \& $p_1$ we have that $-S_n$ is a \textit{Unifromly Most Powerful Test}.
$$p\text{-value}:=p(\x)-\prob(-S_n(\X)\geq-S_n(\x);p_0)=\prob(S_n(\X)\leq S_n(\x);p_0)$$
where $S_n(\X)\sim\text{NegativeBinomial}(n,p_0)$.\\

\remark{Uniformly Most Powerful Tests need not exist}
In general, \textit{Uniformly Most Powerful Tests} need not exist.\\
It might be the case that $(T_1,c_1)$ is best for, say, $\theta_{1,1}\in\Theta_1$.\\
\ie $\forall\ (T',c')$
$$\pi(\theta_{1,1};T_1,c_1)\geq\pi(\theta_{1,1};T',c')$$
but $\exists\ (T_2,c_2)$ st
$$\pi(\theta_{1,2};T_2,c_2)>\pi(\theta_{1,2};T_1,c_1)\quad\theta_{1,2}\in\Theta_1$$
\ie $(T_2,c_2)$ is better than $(T_1,c_1)$.\\

\subsection{Generalised Likelihood Ratio Test}

\remark{Generalised Tests}
In the most general case we would like to test $H_0:\theta\in\Theta_0$ against $H_1:\theta\in\Theta_1$.\\
There is no guarantee of the existence of an optimal test statistic.\\

\proposition{Generalised Likelihood Ratio Test}
We can generalie the likelihood ratio test for simple hypotheses from
$$T_{NP}(\x)=\frac{f_n(\x;\theta_1)}{f_n(\x;\theta_0)}$$
to
$$T_\text{suggested}(\x):=\frac{\sup_{\theta\in\Theta_1}(f_n(\x;\theta)}{\sup_{\theta\in\Theta_0}(f_n(\x;\theta)}$$
\nb The  generalised simple hypothese are $\Theta_i=\{\theta_i\}$ for $\theta_i\in\Theta$.\\

\definition{Likelihood Ratio}
We define a \textit{Likelihood Ratio}
\[\begin{array}{rrl}
\Lambda_n(\x)&:=&{\displaystyle\frac{\sup_{\theta\in\Theta_0}f_n(\x;\theta)}{\sup_{\theta\in\Theta}f_n(\x;\theta)}}\\
&=&{\displaystyle\frac{\sup_{\theta\in\Theta_0}f_n(\x;\theta)}{f_n(\x;\underbrace{\hat\theta_n}_\text{MLE})}}\\
&=&{\displaystyle\min\bigg\{\underbrace{1}_{\hat\theta_n\in\Theta_0},\underbrace{\frac{\sup_{\theta\in\Theta_0}f_n(\x;\theta)}{\sup_{\theta\in\Theta_1}f_n(\x;\theta)}}_{_{\hat\theta_n\not\in\Theta_0}}\bigg\}}
\end{array}\]

\remarkk{Likelihood Ratio}
\begin{enumerate}[label=\roman*)]
	\item The denominiator corresponds to plugging in the \textit{Maximum Likelihood Estimate} in the likelihood (assuming it exists and is unique).
	\item The last equality follows from the fact that $\sup_{\theta\in\Theta}f_n(\x;\theta)\geq\sup_{\theta\in\Theta_0}f_n(\x;\theta)$.\\
	if the inequality is strict then
	$$\sup_{\theta\in\Theta}f_n(\x;\theta)=\sup_{\theta\in\Theta_1}f_n(\x;\theta)>\sup_{\theta\in\Theta_0}f_n(\x;\theta)$$
	and if it is an equality then $\Lambda_n(\x)=1$.
\end{enumerate}

\definition{Nested Parameter Space}
Let $\Theta\subseteq\reals^d$ and define $\phi=(\phi_1,\phi_2):\Theta\to\Phi_1\times\Phi_2$ to a \textit{continuously differentiable bijection} with $\Phi_1\subseteq\reals^r$ and $\Phi_2\subseteq\reals^{d-r}$.\\
$\Theta_0$ is said to be \textit{Nested} within $\Theta$ if for some $c\in\Phi_1\subseteq\reals^r$
$$\Theta_0=\{\theta\in\Theta:\phi_1(\theta)=c\}$$
\nb $\text{dim}(\Theta_0)=d-r$.\\

\example{Nested Parameter Space}
Suppose $\X\iid\text{Normal}(\mu,\sigma^2)$ then $\pmb\Theta=\{\mu,\sigma^2\}$.\\
Suppose $\phi_1(\mu,\sigma^2)=\mu$ \& $\phi_2(\mu,\sigma^2)=\sigma^2$ then $\Phi_1=\reals$ \& $\Phi_2=\reals^+\subseteq\reals$.\\
Then $\Theta_0:=\{(\mu,\sigma^2):\phi_1(\mu,\sigma^2)=0\}$ is \textit{Nested} in $\Theta$.\\

\theorem{Distribution of Test Statistics in Nested Parameter Spaces}
Let $\Theta\subset\reals^d$ and $\X\iid f(\cdot;\theta)$ for $\theta\in\Theta_0$ (\ie $H_0$ is true) where $\Theta_0$ is nested in $\Theta$. Then
$$T_n(\X)=-2\ln\Lambda_n(\X)\to_{\mathcal{D}(\cdot;\theta)}W\sim\chi^2_r$$
with $r=\text{dim}(\Theta)-\text{dim}(\Theta_0)$.\\
\nb The proof is not covered in this course but relies on the taulor expansion of the likelihood.\\

\remark{If value of the null hypothesis is fixed, $H_0:\theta=0$, then $\text{dim}(\Theta_0)=0$}

\example{}
Let $\X\iid\text{Poisson}(\lambda^*)$ for some $\lambda^*>0$.\\
Consider the test
$$H_0:\lambda=\lambda_0\quad\text{against}\quad H_1:\lambda\neq\lambda_0$$
Then
\[\begin{array}{rcl}
T_n(\x)&=&-2\ln\left(\dfrac{f_n(\x;\lambda_0)}{f_n(\x;\hat\lambda_n)}\right)\text{ where }\hat\lambda_n\text{ is MLE of }\lambda\\
&=&-2\ln[\ell_n(\lambda;\x)-\ell_n(\hat\lambda_n;\x)]
\end{array}\]
We know
\[\begin{array}{rrcl}
&f_n(\x;\lambda)&=&{\displaystyle\prod_{i=1}^n\dfrac{e^{-\lambda}\lambda^{x_i}}{x_i!}}\text{ by independence}\\
&&\propto&e^{-n\lambda}\lambda^{n\bar{x}}\\
\implies&\ell_n(\lambda;\x)&=&c-n\lambda+n\bar{x}\ln\lambda\\
\text{Taking}&\ell'_n(\lambda;\x)&=&0\\
\implies&-n+\frac{\bar{x}n}{\lambda}&=&0\\
\implies&\hat\lambda&=&\bar{x}
\end{array}\]
We confirm $\hat\lambda$ is a \textit{Maximum Likelihood Estimate} by showing $\ell''_n(\bar{x};x)<0$.\\
So
\[\begin{array}{rcl}
T_n(\x)&=&-2\left[-n(\lambda_0-\hat\lambda_n)+n\bar{x}\ln\frac{\lambda_0}{\hat\lambda_n}\right]\\
&=&-2\left[-n(\lambda_0-\bar{x})+n\bar{x}\ln\frac{\lambda_0}{\bar{x}}\right]
\end{array}\]
We know that
$$T_n(\X)\sim\chi^2_{1-0}=\chi^2_1\text{ under }H_0$$
since $\text{dim}(\Theta)=1$ and $\text{dim}(\Theta_0)=0$.\\

\definition{Two Sided Test}
A \textit{Two Sided Hypothesis Test} is a hypothesis test where the alternative hypothese covers two separate regions of the parameter space. The general form is
$$H_0:\theta=\theta_0\quad\text{against}\quad H_1:\theta\neq\theta_0$$

\remark{Connection to Confidence Intervals}
Recall that a test of size approximately $\alpha$ is obtained by retain $H_0$ if
$$T_n(\x)=-2\left[\ell_n(\theta_0;\x)-\ell_n(\hat\theta_n;\x)\right]<\chi^2_{r,\alpha}$$
The values of $\theta_0$ which lead to the retention of $H_0$ are
\[\begin{array}{rcl}
C(\x)&=&\left\{\theta:-2[\ell_n(\theta;\x)-\ell_n(\hat\theta_n;\x)]<\chi^2_{r,\alpha}\right\}\\
&=&\left\{\theta:\ell_n(\theta;\x)>\ell_n(\hat\theta_n;\x)-\frac{1}{2}\chi^2_{r,\alpha}\right\}\\
\end{array}\]
This is an asymptotically exact $1-\alpha$ confidence set for $\theta$.

\subsection{Categorical Distributions and Pearson's $\chi^2$ Test}

\definition{Categorical Distribution}
Let $Y$ be a random variable which takes one value from a finite set $\{1,\dots,m\}$ where each value represents a unique category.\\
Let $\text{p}:=(p_1,\dots,p_m)$ be a vector where $p_i=\prob(Y=i)$ then
$$Y\sim\text{Categorical}(\textbf{p})$$
\nb $\sum_{i=1}^mp_i=1,\ p_i\in[0,1]\ \forall\ i\in[1,m]$ and $p$ is a vector of $m-1$ free parameters.\\

\remark{$\text{Bernoulli}(p)\sim\text{Categorical}(1-p,p)$}

\definition{Counts}
Let $Y_1,\dots,Y_n\iid\text{Categorical}(\textbf{p})$.\\
We define the random variable $N_k$ to model the number of times $k$ occurs in a realisation of $Y_1,\dots,Y_n$.
$$N_k:=\sum_{i=1}^n\mathds{1}\{Y_i=k:i\in[1,n]\}$$
This definition gives rise to the random vector
$$\X:=(N_1,\dots,N_m)\sim\text{Multinomial}(n,\textbf{p})$$
with
$$\prob(N_1=n_1,\dots,N_m=n_m;\textbf{p})=\mathds{1}\left\{\sum_{i=1}^mn_i=n\right\}\left\{\dfrac{n!}{\prod_{i=1}^mn_i}\right\}\prod_{i=1}^np_i^{n_i}$$
\nb $\expect(N_i)=np_i$ and $\var(N_i)=np_i(1-p_i)$.

\subsubsection{Generalised Likelihood Ratio Test Statistic}

\definition{Simplex}
TODO Expand this\\
$\mathcal{S}_m$ is a simplex of probability vectors of length $m$ if
$$\mathcal{S}_m:=\left\{(_1,\dots,p_m)\in[0,1]^m:\sum_{i=1}^mp_i=1\right\}$$

\propositionn{Generalised Likelihood Ratio}
$$\Lambda_n(\x)=\frac{f_n(\x;\textbf{p}_0)}{\sup_{\textbf{p}\in\mathcal{S}_m}f_n(\x;\textbf{p})}$$

\theorem{Maximum Likelihood Estimate for Multinomial $\textbf{p}$}
Let $\X=(N_1,\dots,N_m)\sim\text{Multinomial}(n,\textbf{p}^*)$ for some $\textbf{p}^*\in\mathcal{S}_m$ and $\x=(n_1,\dots,n_m)$ be a realisation of $\X$.\\
The maximum likelihood estimate of $\textbf{p}^*$ is
$$\hat{\textbf{p}}(\x)=(\hat{p}_1(\x),\dots,\hat{p}_m(\x))=\left(\frac{n_1},\dots,\frac{n_m}{n}\right)$$

\proof{Theorem 8.1}
Note that
$$\sum_{i=1}^mp_i=1\implies p_m=1-\sum_{i=1}^{m-1}p_i$$
Hence there are only $m-1$ independent variables and
\[\begin{array}{rcl}
L(\textbf{p},\x)&=&L(p_1,\dots,p_{m-1};\x)\\
&\propto&{\displaystyle \prod_{j=1}^mp_j^{n_j}}\\
&=&{\displaystyle\left(\prod_{j=1}^{m-1}p_j^{n_j}\right)\left(1-\sum_{i=1}^{m-1}p_i\right)^{n_m}}
\end{array}\]
So
$$\ell(p_1,\dots,p_{m-1};\x)=C+\left(\sum_{i=1}^{m-1}n_j\ln p_j\right)+n_m\ln\left(1-\sum_{i=1}^{m-1}p_i\right)$$
Now for $k=1,\dots,m-1$.\\
\[\begin{array}{rrcl}
\text{Setting}\frac{\partial}{\partial p_k}\ell(p_1,\dots,p_{m-1};\x)&=&\frac{n_k}{p_k}-\frac{n_m}{1-\sum_{i=1}^{m-1}p_i}\\
&&=&0\\
\implies&\frac{n_k}{p_k}&=&\frac{n_m}{p_m}\ \forall\ k\in[1,m]
\end{array}\]
So $\frac{n_1}{p_1}=\dots=\frac{n_m}{p_m}=c$ and $\sum_{i=1}^mp_i=1$.\\
$$\implies\sum_{i=1}^m\frac{n_i}{c}=1\implies\sum_{i=1}^mn_i=c\implies n=c$$
Hence $\frac{n_k}{p_k}=n\implies\hat{p}_j=\frac{n_k}{n}\ \forall\ k\in[1,m]$.\\
In order to confirm that this is a maximum we will show that $\ell(\textbf{p};\x)$ is concave.\\
\ie for $\lambda\in[0,1]\ \ell(\lambda \textbf{p}+(1-\alpha)\textbf{p}';\x)\geq\lambda\ell(\textbf{p};\x)+(1-\lambda)\ell(\textbf{p}';\x)$.
\[\begin{array}{rcl}
\ell(\lambda\textbf{p}+(1-\lambda)\textbf{p}';\x)&=&\sum_{i=1}^m n_i\ln(\lambda p_i+(1-\lambda)p_i')\\
&\geq&\sum_{i=1}^mn_i\left[\lambda_i\ln p_i + (1-\lambda)\ln p_i'\right]\text{ since $\ln x$ is concave}\\
&=&\lambda\sum_{i=1}^mn_i\ln p_i+n_i(1-\lambda)\ln p'_i\\
&=&\lambda\ell(\textbf{p};\x)+(1-\lambda)\ell(\textbf{p}';\x)
\end{array}\]
Thus concave.\\
It follows that
$$\Lambda_n(\x)=\frac{f_n(\x;\textbf{p}_0)}{\sup_{\textbf{p}\in\mathcal{S}_m}f_n(\x;\textbf{p})}=\prod_{i=1}^m\frac{p_{0,i}^{n_i}}{\hat{p}_i^{n_i}}=\prod_{i=1}^m\frac{p_{0,i}^{n_i}}{(n_i/n)^{n_i}}$$
so that
$$T_n(\x)=-2\ln\Lambda_n(\x)=-2\sum_{i=1}^mn_i\{\ln p_{0,i}-\ln(n_i/n)\}$$
is the \textit{Generalised Likelihood Ratio} test statistic. From the general theorem
$$T_n(\x)\to_{\mathcal{D}(\cdot;\textbf{p}_0}\chi^2_{m-1}$$
since $\text{dim}(\mathcal{S}_m)=m-1$.\\
Many people rewrite this statistic as
\[\begin{array}{rcl}
T_n(\x)&=&{\displaystyle2\sum_{j=1}^mo_j\ln\left(\frac{0_j}{e_j}\right)}\\
&=&{\displaystyle2\sum_{j=1}^mn_j\ln\left(\frac{n_j/n}{p_{0,j}}\right)}\\
&=&{\displaystyle-2\sum_{j=1}^mn_j\ln\left(\frac{n_j}{np_{0,j}}\right)}
\end{array}\]
where $o_j=n_j$ is the observered number in category $j$ and $e_j=np_{0,j}$ is the expected number in category $j$.$\hfill\square$

\subsubsection{Pearson's $\chi^2$ Test Statistic}

\definition{Pearson's $\chi^2$ Test Statistic}
Let $\x\sim\text{Categoritcal}(\textbf{p})$ where $\textbf{p}:=(p_0,\dots,p_m)$ and $\x$ is a relisation of $\X$.\\
We define \textit{Pearson's $\chi^2$ Test Statistic}  as
$$T_\text{Pearson}(\x):=\sum_{j=1}^m\frac{(n_j-np_{j})^2}{np_{j}}=\sum_{j=1}^m\frac{(o_j-e_j)^2}{e_j}\to_{\mathcal{D}(\cdot;\textbf{p})}\chi^2_{m-1}$$
where $o_j$ is the number of observations of category $j$ and $e_j$ is the expected number of observations of category $j$.
\nb TODO - something about degrees of freedom.\\

\subsection{Goodness-of-Fit Example - Mendel's Peas}
Gregor Mendel, the father of modern genetics, performed experiments to test his theory of inheritance in the 1850s \& 60s.\\
There are $k=4$ types of peas
\begin{enumerate}
	\item Round Yellow.
	\item Wrinked Yellow.
	\item Round Green, and
	\item Wrinkled Green,
\end{enumerate}
According to Medel's theory of inheritance, the number of peas of each type in the second generation of a crossing experiment should be an observatio of a $\text{Multinomial}(n,\textbf{p}_0)$ random vector where
$$\textbf{p}_0=\left(\frac{9}{16},\frac{3}{16},\frac{3}{16},\frac{1}{16}\right)$$
This consitutes the null hypothesis.\\
In Mendel's experiment, with $n=556$, he recored
$$\x=(n_1,\dots,n_4)=(315,101,108,32)$$
The test statistic is
\[\begin{array}{rcl}
w&=&\displaystyle\sum_{i=1}^k\frac{(n_i-np_{0,i})^2}{np_{0,i}}\\
&=&0.47
\end{array}\]
The $p$-value is therefore the probability that a $\chi^2_{k-1}=\chi^2_3$ random variable exceeds $0.47$
$$p(\x)=\prob(\chi^2_3>0.47)=0.925$$
We would not reject the null hypothesis for any reasonable $\alpha$, so the data does not contadict his theory.
In fact the $P$-value is quite large and there is some controversy regarding whether Mendel's results are \textit{too good}.\\

Ronald Fisher looked at Mendel's data for a sequence of $m$ similar experiments and discovered that they seemed all to obtain this kind of unusually good agreement. Each test statistic $w_i$ is modelled approximately as an observation of an independent $\chi^2_{k_i-1}$ random variable where $k_i$ is eh number of possible outcomes in experiment $i$.\\
Since the sum of independent $\chi^2_{k_i-1}$ random variables is $\chi^2_{\sum_i(k_i-1)}$ random variable, one can use the sum of the test statistics as a test statistic and compute the $p$-value
$$p(\x)=\prob\left(Y\geq\sum_{i=1}^mw_i\right)\text{ where }Y\sim\chi^2_{\sum_i(k_1-1)}$$
The fact he obtaind a \textit{pooled $\chi^2$} test statistic of $\sum_{i=1}^mw_i=42$ and a a value of $\sum_{i=1}^m(k_i-1)=84$.\\
$$p(\x)=\prob(Y\geq42)\approx0.999965\text{ where }Y\sim\chi^2_{84}$$
But Fisher's $H_0$ was that Mendel's data was \textit{gathered honestly} and assumed that Mendel's theory is true. And his alternative hypthesis $H_1$ was that the data was \underline{not} collected honestly. Thus his $p$-value is in fact
$$\prob(-Y\geq-42)=\prob(Y\leq42)=0.000035$$
So Fisher's $H_0$ is rejected with a very low $\alpha$.\\
Therea are many who belive taht neither Mendel, nor his assistants, did anything untoward in their experiments but it is an interesting applicaiton nonetheless.

\newpage
\setcounter{section}{-1}
\section{Appendix}

\definition{Gradient}
$$\nabla f(\pmb{\theta};\x):=\left(\dfrac{\partial f(\pmb{\theta};\x)}{\partial\theta_1},\dots,\dfrac{\partial f(\pmb{\theta};\x)}{\partial\theta_n}\right)$$

\definition{Hessian}
$$\nabla f(\pmb{\theta};\x):=\begin{pmatrix}
\dfrac{\partial^2 f(\pmb{\theta};\x)}{\partial\theta_1^2}&\dfrac{\partial^2 f(\pmb{\theta};\x)}{\partial\theta_1\partial\theta_2}&\dots&\dfrac{\partial^2 f(\pmb{\theta};\x)}{\partial\theta_n\theta_1}\\
\dfrac{\partial^2 f(\pmb{\theta};\x)}{\partial\theta_1^2}&\dfrac{\partial^2 f(\pmb{\theta};\x)}{\partial\theta_1\theta_2}&\dots&\dfrac{\partial^2 f(\pmb{\theta};\x)}{\partial\theta_n\theta_2}\\
\vdots&\vdots&\ddots&\vdots\\
\dfrac{\partial^2 f(\pmb{\theta};\x)}{\partial\theta_1\partial\theta_n}&\dfrac{\partial^2 f(\pmb{\theta};\x)}{\partial\theta_2\theta_n}&\dots&\dfrac{\partial^2 f(\pmb{\theta};\x)}{\partial\theta_n^2}
\end{pmatrix}$$

\subsection{Notation}

\begin{tabular}{|l|l|}
\hline
Notation&Denotes\\
\hline
$Z_n\to_{\prob}Z$&$\{Z_n\}_{n\in\nats}$ converges in \textit{Probabilitiy} to random variable $Z$.\\
$Z_n\to_\mathcal{D}Z$&$\{Z_n\}_{n\in\nats}$ converges in \textit{Distribution} to random variable $Z$.\\
$Z_n\to_{qm}Z$&$\{Z_n\}_{n\in\nats}$ converges in \textit{Quadratic Mean} to random variable $Z$.\\
$\theta\in\Theta\subseteq\reals^{d_\theta}$&Scalar or vector parameter characterising a probability distribution\\
$\hat{\theta}$&Estimation for the value of the parameter $\theta$\\
$\theta^*$&True value of the paramter $\theta$\\
$\prob$&Probability measure $\prob:\mathcal{F}\to[0,1]$\\
$\Omega$&Sample space\\
$X$&Scalar random variable\\
$\mathcal{F}$&Sigma field (Set of events)\\
$\chi$&Support of rv $XX$. A set set $X$ is definitely in it \ie $\prob(X\in\chi;\theta)=1$\\
$\X$&Vector consiting of scalar random variables\\
\hline
\end{tabular}

\subsection{R}
\begin{tabular}{|l|l|}
\hline
Command&Result\\
\hline
\textit{hist(a)}&Plots a histogram of the values in array $a$\\
\textit{mean(a)}&Returns the mean value of array $a$\\
\textit{rbinom(s,n,p)}&Samples $n$ of $Bi(n,p)$ random variables\\
\textit{rep(v,n)}&Produces an array of size $n$ where each entry has value $v$\\
$x\leftarrow v$&Maps value $v$ to variable $x$\\
\hline
\end{tabular}

\subsection{Probability Distributions}

\definition{Binomial Distribution}
Let $X$ be a discrete random variable modelled by a \textit{Binomial Distribution} with $n$ events and rate of success $p$.\\
\[\begin{array}{rcl}
p_X(k)&=&{n\choose k}p^k(1-p)^{n-k}\\
\expect(X)np=&\&&Var(X)=np(1-p)
\end{array}\]

\definition{Gamma Distribution}
Let $T$ be a continuous randmo variable modelled by a \textit{Gamma Distribution} with shape parameter $\alpha$ \& scale parameter $\lambda$. Then
\[\begin{array}{rcll}
f_T(x)&=&\dfrac{\lambda^\alpha x^{\alpha-1}e^{-\lambda x}}{\Gamma(\alpha)}&\mathrm{for\ }x>0\\
\expect(T)=\dfrac{\alpha}{\lambda}&\&&Var(T)=\dfrac{\alpha}{\lambda^2}
\end{array}\]
\nb $\alpha,\lambda>0$.\\

\definition{Exponential Distribution}
Let $T$ be a continuous random variable modelled by a \textit{Exponential Distribution} with parameter $\lambda$. Then
\[\begin{array}{rcl}
f_T(t)&=&\indicator\{t\geq0\}.\lambda e^{-\lambda t}\\
F_T(t)&=&\indicator\{t\geq0\}.\left(1-e^{-\lambda t}\right)\\
\expect(X)=\frac{1}{\lambda}&\&&Var(X)=\frac{1}{\lambda^2}
\end{array}\]
\nb Exponential Distribution is used to model the wait time between decays of a radioactive source.\\

\definition{Normal Distribution}
Let $X$ be a continuous random variable modelled by a \textit{Normal Distribution} with mean $\mu$ \& variance $\sigma^2$.\\
Then
\[\begin{array}{rcl}
f_X(x)&=&\dfrac{1}{\sqrt{2\pi\sigma^2}}e^{-\frac{(x-\mu)^2}{2\sigma^2}}\\
F_X(x)&=&\dfrac{1}{\sqrt{2\pi\sigma^2}}\int\limits_{-\infty}^xe^{-\frac{(y-\mu)^2}{2\sigma^2}}dy\\
M_X(\theta)&=&e^{\mu\theta+\sigma^2\theta^2(1/2)}\\
\expect(X)=\mu&\&&Var(X)=\sigma^2
\end{array}\]

\definition{Poisson Distribution}
Let $X$ be a discrete random variable modelled by a \textit{Poisson Distribution} with parameter $\lambda$. Then
\[\begin{array}{rcll}
p_X(k)&=&\dfrac{e^{-\lambda}\lambda^k}{k!}&\mathrm{For\ }k\in\nats_0\\
\expect(X)=\lambda&\&&Var(X)=\lambda
\end{array}\]
\nb Poisson Distribution is used to model the number of radioactive decays in a time period.\\


\end{document}
